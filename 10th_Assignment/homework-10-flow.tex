\documentclass[12pt,a4]{article}




\usepackage{graphicx,amsmath,amssymb,amsthm, boxedminipage,xcolor}

\usepackage{algorithm}
\usepackage{algpseudocode}


\newtheorem{theorem}{Theorem}[section]
\newtheorem{proposition}[theorem]{Proposition}
\newtheorem{lemma}[theorem]{Lemma}
\newtheorem{corollary}[theorem]{Corollary}
\newtheorem{definition}[theorem]{Definition}

\newtheorem*{theorem*}{Theorem}
\newtheorem*{lemma*}{Lemma}
\newtheorem*{proposition*}{Proposition}


\newtheorem{exercise}[theorem]{Exercise}
\newtheorem{exerciseD}[theorem]{*Exercise}
\newtheorem{exerciseDD}[theorem]{**Exercise}

\let\oldexercise\exercise
\renewcommand{\exercise}{\oldexercise\normalfont}


\newcommand{\E}{\mathbb{E}}
\newcommand{\scalar}[2]{\ensuremath{\langle #1, #2\rangle}}
\newcommand{\floor}[1]{\left\lfloor #1 \right\rfloor}
\newcommand{\ceil}[1]{\left\lceil #1 \right\rceil}
\newcommand{\norm}[1]{\|#1\|}
\newcommand{\pfrac}[2]{\left(\frac{#1}{#2}\right)}
\newcommand{\nth}[1]{#1^{\textsuperscript{th}}}
\newcommand{\core}{\textnormal{core}}





\newcommand{\poly}{\textnormal{poly}}
\newcommand{\quasipol}{\textnormal{quasipol}}
\newcommand{\ssubexp}{\textnormal{stronglySubExp}}
\newcommand{\wsubexp}{\textnormal{weaklySubExp}}
\newcommand{\simplyexp}{\textnormal{E}}
\newcommand{\expo}{\textnormal{Exp}}



\newcommand{\N}{\mathbb{N}}
\newcommand{\nn}{\mathbb{N}_0^n}
\newcommand{\R}{\mathbb{R}}
\newcommand{\Z}{\mathbb{Z}}


\definecolor{darkgreen}{rgb}{0,0.6,0}


\date{}

\title{
  Mathematical Foundations \\of \\Computer Science\\
  \vspace{3mm}
{\normalsize CS 499,	Shanghai Jiaotong University,  Dominik Scheder}
}

\begin{document}

\maketitle

%\begin{quotation}
%  You are welcome to discuss the exercises in the discussion
%  forum. Please take them serious. Doing the exercises is as important
%  than watching the videos.
%
%  I intentionally included very challenging exercises and marked them
%  with one or two ``$*$''. No star means you should be able to solve
%  the exercises without big problems once you have understood
%  the material from the video lecture. One star means it requires 
%  significant additional thinking. Two stars means it is not 
%  unlikely that you will fail to solve them, even once you have understood
%  the material and thought a lot about the exercise. Don't feel bad
%  if you fail. Failure is part of learning.
%
%  This is the first time this course is online. Thus there might be mistakes
%  (typos or more serious conceptual mistakes) in the exercises. I will be 
%  grateful if you point them out to me!
%\end{quotation}





\setcounter{section}{9}

\section{Network Flow}



\begin{itemize}
 \item Homework assignment published on Monday 2018-05-07
 \item Submit questions and first solution by Sunday, 2018-05-13, 12:00
 \item Submit final solution by Sunday, 2018-05-20.
\end{itemize}


\begin{exercise}[From the video lecture]
   Recall the definition of the value of a flow: ${\rm val}(f) = \sum_{v \in V} f(s,v)$.
   Let $S \subseteq V$ be a set of vertices that contains $s$ but not $t$. Show that
   \begin{align*}
         {\rm val}(f) = \sum_{u \in S, v \in V \setminus S} f(u,v) \ .
   \end{align*}
   That is, the total amount of flow leaving $s$ equals the total amount of flow
   going from $S$ to $V \setminus S$.
   \textbf{Remark.} It sounds obvious. However, find a formal proof that works with the
   axiomatic definition of flows.
\end{exercise}
\begin{proof}By flow conservation, $\sum_{w}f(v,w)=0$, for all $v\in V\setminus\{s,t\}$.\\
Also by definition of the value of a flow, we have:\\
\begin{align*}
{\rm val}(f) &= \sum_{w \in V} f(s,w)\\
&=\sum_{u\in S, w\in V}f(u,w)\\
& =\sum_{u \in S}(\sum_{w_1\in S}f(u,w_1)+\sum_{w_2\in V\setminus S}f(u,w_2))\\
&=\sum_{u \in S, w_1\in S}f(u,w_1)+\sum_{u \in S, w_2\in V\setminus S}f(u,w_2)\\
&= \sum_{u \in S , v \in V \setminus S} f(u,v)+\sum_{u \in S, w_1\in S}f(u,w_1)
%&=\sum_{u \in S , v \in V \setminus S} f(u,v) \\
\end{align*}
Recall the skew-symmetry that $f(u,v)=-f(v,u)$, we have:
\begin{equation}
\sum_{u \in S, w_1\in S}f(u,w_1)=\sum_{w_1\in S, u \in S}-f(w_1,u)=-\sum_{u \in S, w_1\in S}f(u,w_1)=0
\end{equation}
In conclusion, ${\rm val}(f) =\sum_{u \in S, v \in V \setminus S} f(u,v) \ .$
\end{proof}

\begin{exercise}
Let $G = (V,E,c)$ be a flow network.
  Prove that flow is ``transitive'' in the following sense:
  If there is a flow from $s$ to $r$ of value $k$,
  and a flow from $r$ to $t$ of value $k$, then
  there is a flow from $s$ to $t$ of value $k$.
  \textbf{Hint.} The solution is extremely short. If you are trying
  something that needs more than 3 lines to write, you are on the wrong
  track.
\end{exercise}
\begin{proof}
Suppose there is no flow from $s$ to $t$ of value $k$. Then there is a s-t cut $s\in S, t\in V\setminus S$ such that $c(S,V\setminus S)<k$. If $r\in S$, then there is a r-t cut $(S,V\setminus S)$ such that the capacity is less than $k$, so there's no flow of value $k$ from $r$ to $t$. Else if $r\in V\setminus S$, then there is a s-r cut $(S, V\setminus S)$, whose capacity is less than $k$ and no flow value of $k$ exists. In either of these cases, we reach a contradiction, so flow must be transitive.
\end{proof}



\subsection{An Algorithm for Maximum Flow}

Recall the algorithm for Maximum Flow presented in the video. It is usually called
the Ford-Fulkerson method.

\begin{algorithm}[h]
\caption{Ford-Fulkerson Method}\label{algorithm-encoding}
\begin{algorithmic}[1]
\Procedure{FF}{$G=(V, E), s, t, c$}
  \State Initialize $f$ to be the all-$0$-flow.
   \While{there is a path $p$ form $s$ to $t$ in the residual network $G_{f}$}
    \State $c_{\rm min} := \min \{ c_f(e) \ | \ e \in p \}$
    \State let $f_p$ be the flow in $G_f$ that routes $c_{\rm min}$ flow along $p$
    \State $f:= f + f_p$
   \EndWhile
   \State \texttt{//} now $f$ is a maximum flow
  \State $S := \{v \in V \ | \ G_f \textnormal{ contains a path from } s \textnormal{ to } v \}$
  \State \texttt{//} $S$ is a minimum cut
  \State \Return $(f,S)$
\EndProcedure
\end{algorithmic}
\end{algorithm}

We proved in the lecture that $f$ is a maximum flow and $S$ is a minimum cut,
by showing that upon termination of the while-loop, ${\rm val}(f) = {\rm cap}(S)$.
The problem is that the while-loop might not terminate. In fact, there is an example
with capacities in $\mathbb{R}$ for which the while loop does not terminate, and the
value of $f$ does not even converge to the value of a maximum flow. As indicated
in the video, a little twist fixes this:
\begin{quotation}
  \textbf{Edmonds-Karp Algorithm:} Execute the above
  Ford-Fulkerson Method, but in every iteration choose $p$ to be a
  shortest $s$-$t$-path in $G_f$. Here, ``shortest'' means minimum number of edges.
\end{quotation}
In a series of exercises, you will now show that this algorithm always terminates
after at most $n \cdot m$ iterations of the while loop (here $n = |V|$ and $m = |E|$).

\begin{definition}
   Let $(G,s,t,c)$ be a flow network and $k \in \mathbb{N}_0$.
   A {\em $k$-layering} is a partition of $V = V_0 \cup \dots \cup V_k$ such that
   (1) $s \in V_0$, (2) $t \in V_k$, (3)
   for every edge $(u,v) \in E$ the following holds: suppose $u \in V_i$ and $v \in V_j$.
   Then $j \leq i+1$.
   In words, point (3) states that every edge moves at most one level forward.
\end{definition}

The figure below illustrates this concept: for one network we show two possible layerings
and something that looks like a layering but is not:
\begin{center}
\includegraphics[width=\textwidth]{figures/layering.pdf}
\end{center}

\begin{exercise}
   Suppose the network $(G,s,t,c)$ has a $k$-layering. Show that ${\rm dist}(s,t) \geq k$.
   That is, every $s$-$t$-path in $G$ has at least $k$ edges.
\end{exercise}
\begin{proof} Prove by contradiction:\\
Suppose that there is a $s$-$t$-path, say $s$-$v_1$-$\dots$-$v_{n}$-$t$, in $G$ that has less than $k$ edges ($n+1<k$). By definition, $s\in V_0$ and $t\in V_k$ and every edge moves at most one level forward. Hence, for the $s$-$t$-path, $s$-$v_1$-$\dots$-$v_{n}$-$t$, we have:
\begin{align*}
&s\in V_0\\
&v_1\in V_{a_1}, a_1\le 1\\
&v_2\in V_{a_2}, a_2\le a_1+1\le2\\
\dots
&v_n\in V_{a_n}, a_n\le a_{n-1}+1\le n\\
&t\in V_{a_{n+1}},a_{n+1}\le a_n+1\le n+1<k
\end{align*}
However, $t\in V_k$, a contradiction.
\end{proof}

\begin{exercise}
   Conversely, suppose ${\rm dist}(s,t) = k$. Show that $(G,s,t,c)$ has a $k$-layering.
\end{exercise}
\begin{proof}
Prove by construction:\\
${\rm dist}(s,t) = k$, suppose the shortest path between $s$ and $t$ are $s$-$v_1$-$\dots$-$v_{k-1}$-$t$. Construction steps are as follows:
\begin{itemize}
\item First, add $s$ to $V_0$, $v_1$ to $V_1$, \dots, $v_{k-1}$ to $V_{k-1}$, $t$ to $V_k$.
\item Consider the vertices that are directly connected with vertices that are already in the partitions, iteratively add them to any partitions as long as the adding satisfies \emph{POINT 3} above.
\item Recursively do step 2 above until no vertices left (since a flow network is a connected component as a whole).
\end{itemize}

\end{proof}

Let $(G,s,t,c)$ be a flow network and $V_0,\dots,V_k$ a $k$-layering. We call this
layering {\em optimal} if ${\rm dist}_G(s,t) = k$. Here, ${\rm dist}_G(u,v)$
is the shortest-path distance from $s$ to $t$ (measured by number of edges).
If there is no path from $s$ to $t$, we set ${\rm dist}_G(s,t) = \infty$. In this case,
no layering is optimal.
For example, the $3$-layering
in the above figure is optimal, but the $1$-layering in the middle of the above figure
is not.
Let us explore how layerings and the Ford-Fulkerson Method interact.

\begin{exercise}
   Let $(G,s,t,c)$ be a flow network and $V_0, V_1, \dots, V_k$ be an optimal layering
   (that is, $k = {\rm dist}_G(s,t)$.
   Let $p$ be a path from $s$ to $t$ of length $k$.
   Suppose we route some flow $f$ along $p$ (of some
   value $c_{\rm min} > 0$) and let $(G_f,s,t,c_f)$ be the residual network. Show that
   $V_0, V_1,\dots, V_k$ is a layering of $(G_f,s,t,c_f)$, too. Obviously, condition (1) and (2) in
   the definition of $k$-layerings still hold, so you only have to check  condition (3).
\end{exercise}
\begin{proof}
Consider the route operation, changes made on original edges in $G$ are only limited on the capacity value and the added new edges each has a corresponding reversely directed edge in $G$, which is to say, that if any edge in $G$ satisfied \emph{POINT 3}, then any edge in $G_f$ will not violate \emph{POINT 3}.
\end{proof}


\begin{exercise}
   Show that every network $(G,s,t,c)$ has an optimal layering, provided there is a path
   from $s$ to $t$.
\end{exercise}
\begin{proof}
Since there is a path from $s$ to $t$, there is a shortest path, say the $dist(s,t)=k$. According to \textbf{Exercise 10.5}, $G$ has a $k-layering$, which is the optimal layering by definition.
\end{proof}
\begin{exercise}
   Imagine we are in some iteration of the while-loop of the Ford-Fulkerson method.
   Let $V_0, \dots, V_k$ be an optimal layering of $(G,s,t,c)$. Show that after at most $m$
   iterations of the while-loop, $V_0,\dots,V_k$ ceases
   to be an optimal layering. \textbf{Remark.} Note that it is the {\em network} that changes from
   iteration to iteration of the while-loop, not the partition $V_0,\dots,V_k$. We consider
   the partition $V_0,\dots,V_k$ to be fixed in this exercise.
\end{exercise}
\begin{proof}
  Derived from the difinition of the layering. any flow $f$ can not pass the partition forwarding, which means that flow go from $V_i$ to $V_j$($j > i + 1$) directly is not permitted. Because there is at most $m$ edges from $V_i$ to $V_{i+1}$, after at most $m$ iterations, if more flows want to pass the partition, they need to go back from $V_{i+1}$ to $V_i$, making $dist(s,t)>k$ and $V_0, \dots, V_k$ no longer the optimal layering.
\end{proof}

\begin{exercise}
Show that the Edmonds-Karp algorithm terminates after $n \cdot m$ iterations of the
while-loop. \textbf{Hint.} Initially, compute an optimal $k$-layering (which?). Then keep
this layering as long as its optimal. Once it ceases to be optimal, compute a new optimal
layering. Note that the Edmonds-Karp algorithm does not actually need to compute any
layering. It's us who compute it to show that $n \cdot m$ bound on the number of iterations.
\end{exercise}
\begin{proof}
  For the Edmonds-Karp algorithm is a special cases of the Ford-Fulkerson algorithm, when using Edmonds-Karp algorithm, after at most $m$ iterations of the while-loop, $V_0,\dots,V_k$ will cease to be an optimal layering. That is, $dist(s,t)$ is no longer $k$.

  Because the Edmonds-Karp always choose the short path from $s$ to $t$ as $p$, then $dist(s,t) = k'$ is now at least $k+1$. And we have $1 \leq k \leq n$, we can find a new optimal layering at most $n$ times.

  Hence the Edmonds-Karp algorithm terminates after $n \cdot m$ iterations of the
  while-loop.
\end{proof}

\begin{exercise}
   Show that every network has a maximum flow $f$.
   That is, a flow $f$ such that ${\rm val}(f) \geq {\rm val}(f')$ for every flow $f'$.
   \textbf{Remark.} This sounds obvious but it is not. In fact, there might be an infinite
   sequence of flows $f_1, f_2, f_3, \dots$ of increasing value that does not reach any maximum.
    Use the previous exercises!
\end{exercise}
\begin{proof}
  Using the Edmonds-Karp algorithm, the iteration with be terminated in at most $n \cdot m$ iterations, thus will never producing a infinite sequence. That is, the Edmonds-Karp algorithm will find a maximum flow $f$.
\end{proof}


\end{document}

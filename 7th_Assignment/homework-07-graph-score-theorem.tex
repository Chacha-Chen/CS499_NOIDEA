\documentclass[12pt,a4]{article}




\usepackage{graphicx,amsmath,amssymb,amsthm, boxedminipage,xcolor}

\usepackage{algorithm}
\usepackage{algpseudocode}


\newtheorem{theorem}{Theorem}[section]
\newtheorem{proposition}[theorem]{Proposition}
\newtheorem{lemma}[theorem]{Lemma}
\newtheorem{corollary}[theorem]{Corollary}
\newtheorem{definition}[theorem]{Definition}

\newtheorem*{theorem*}{Theorem}
\newtheorem*{lemma*}{Lemma}
\newtheorem*{proposition*}{Proposition}


\newtheorem{exercise}[theorem]{Exercise}
\newtheorem{exerciseD}[theorem]{*Exercise}
\newtheorem{exerciseDD}[theorem]{**Exercise}

\let\oldexercise\exercise
\renewcommand{\exercise}{\oldexercise\normalfont}


\newcommand{\E}{\mathbb{E}}
\newcommand{\scalar}[2]{\ensuremath{\langle #1, #2\rangle}}
\newcommand{\floor}[1]{\left\lfloor #1 \right\rfloor}
\newcommand{\ceil}[1]{\left\lceil #1 \right\rceil}
\newcommand{\norm}[1]{\|#1\|}
\newcommand{\pfrac}[2]{\left(\frac{#1}{#2}\right)}
\newcommand{\nth}[1]{#1^{\textsuperscript{th}}}
\newcommand{\core}{\textnormal{core}}





\newcommand{\poly}{\textnormal{poly}}
\newcommand{\quasipol}{\textnormal{quasipol}}
\newcommand{\ssubexp}{\textnormal{stronglySubExp}}
\newcommand{\wsubexp}{\textnormal{weaklySubExp}}
\newcommand{\simplyexp}{\textnormal{E}}
\newcommand{\expo}{\textnormal{Exp}}



\newcommand{\N}{\mathbb{N}}
\newcommand{\nn}{\mathbb{N}_0^n}
\newcommand{\R}{\mathbb{R}}
\newcommand{\Z}{\mathbb{Z}}


\definecolor{darkgreen}{rgb}{0,0.6,0}


\date{}

\title{
  Mathematical Foundations \\of \\Computer Science\\
  \vspace{3mm}
{\normalsize CS 499,	Shanghai Jiaotong University,  Dominik Scheder}
}

\begin{document}

\maketitle

%\begin{quotation}
%  You are welcome to discuss the exercises in the discussion
%  forum. Please take them serious. Doing the exercises is as important
%  than watching the videos.
%
%  I intentionally included very challenging exercises and marked them
%  with one or two ``$*$''. No star means you should be able to solve
%  the exercises without big problems once you have understood
%  the material from the video lecture. One star means it requires 
%  significant additional thinking. Two stars means it is not 
%  unlikely that you will fail to solve them, even once you have understood
%  the material and thought a lot about the exercise. Don't feel bad
%  if you fail. Failure is part of learning.
%
%  This is the first time this course is online. Thus there might be mistakes
%  (typos or more serious conceptual mistakes) in the exercises. I will be 
%  grateful if you point them out to me!
%\end{quotation}





\setcounter{section}{6}

\section{The Graph Score Theorem}



\begin{itemize}
 \item Homework assignment published on Monday, 2018-04-09.
 \item Submit questions and first solution by Sunday, 2018-04-15, 12:00 by
 email to dominik.scheder@gmail.com and the TAs.
 \item You will receive feedback by Wednesday, 2018-04-18.
 \item Submit your final solution by Sunday, 2018-04-22 to me and the two TAs.
\end{itemize}



\begin{exercise}
  Describe, in simple sentences with a minimum of mathematical formalism, (1) the score
  of a graph, (2) what the graph score theorem is, (3) the idea of the 
  graph score algorithm, (4) where the difficult part of its proof is.
  Imagine you have a friend who does not take this class, and think about how to answer
  the above questions to them.
\end{exercise} 



\subsection{Alternative Graphs}

Now we will look at different notions of graphs. As defined in class and in the video
lectures, a graph is a pair $G = (V,E)$ where $V$ is a (usually finite) set, called the {\em vertices},
and $E \subseteq {V \choose 2}$, called the set of {\em edges}.

\paragraph{Multigraphs.} 
A {\em multigraph} is like a graph, but you can have several parallel edges between
two vertices. You cannot, however, have self-loops. That is, there cannot
be an edge from $u$ to $u$ itself. This is an example of a multigraph:

\begin{center}
  \includegraphics[width=0.25\textwidth]{figures/multigraph-score.pdf}
\end{center}

We can define degree and score for multigraphs, too. For example, this multigraph
has score $(4,4,2)$. Obviously no graph
can have this score.



\begin{exercise}
  State a score theorem for multigraphs. That is, something like
  \begin{theorem}[Multigraph Score Theorem]
     Let $(a_1,\dots,a_n) \in \N_0^n$. There is a multigraph
     with this score if and only if \texttt{\textup{<fill in some simple criterion here>}}.
  \end{theorem}
  
  
  \textbf{Remark.} This is actually
  simpler than for graphs.
\end{exercise}

\begin{exercise}
  Prove your theorem.
\end{exercise}


\newcommand{\wdeg}{\textnormal{wdeg}}

\paragraph{Weighted graphs.} A weighted graph is a graph in which every edge $e$ has a non-negative weight $w_e$.
In such a graph the {\em weighted degree} of a vertex $u$ is $\wdeg(u) = \sum_{ \{u,v\} \in E} w_{\{u,v\}}$.

\begin{center}
  \includegraphics[width=0.25\textwidth]{figures/weighted-graph-score.pdf}
\end{center}

This is an example of a weighted graph, which has score $(3,2,2)$. Obviously no graph
and no multigraph can have this score.

\begin{exercise}
  State a score theorem for weighted graphs. That is, state something like
  \begin{theorem}[Weighted Graph Score Theorem]
     Let $(a_1,\dots,a_n) \in \R_0^n$. There is a weighted graph
     with this score if and only if \texttt{\textup{<fill in some simple criterion here>}}.
  \end{theorem}
  \textbf{Remark.} This 
  is actually even simpler.
\end{exercise}


\begin{exercise}
  Prove your theorem.
\end{exercise}

\paragraph{Allowing negative edge weights.} Suppose now we allow negative edge weights, like here:
\begin{center}
  \includegraphics[width=0.25\textwidth]{figures/arbitrary-weighted-graph-score.pdf}
\end{center}
This ``graph with real edge weights'' has score $(2,0,0)$. This score is impossible for
graphs, multigraphs, and weighted graphs with non-negative edge weights.


\begin{exercise}
  State a score theorem for weighted graphs when we allow
  negative edge weights. That is, state a theorem like
  \begin{theorem}[Score Theorem for Graphs with Real Edge Weights]
     Let $(a_1,\dots,a_n) \in \R^n$. There is a  graph with real edge weights
     with this score if and only if \texttt{\textup{<fill in some simple criterion here>}}.
  \end{theorem}  
\end{exercise}

\begin{exercise}
Prove your theorem.
\end{exercise}


\begin{exercise}
   For each student ID $(a_1,\dots,a_n)$ in your group, check whether 
   this is (1) a graph score, (2) a multigraph score, (3) a weighted graph score, or
   (4) the score of a graph with real edge weights.
   
   Whenever the answer is {\em yes}, show the graph, when it is {\em no}, 
   give a short argument why. 
\end{exercise}

\paragraph{Example Solution.} My work ID is 50411. 
This is a weighted graph score, as shown by this picture:
\begin{center}
\includegraphics[width=0.4\textwidth]{figures/graph-50411.pdf}
\end{center}
This settles (3). It is not a multigraph score, because BLABLABLA. I won't give more details, as it might
give too many hints about Exercise 7.2. Alright, this settles (2). Note that I do {\em not} need to answer 
(4), as this is already answered by (3). Neither do I need to answer (1), as a ``no'' for (2) implies
a ``no'' for (1).






\end{document}

\documentclass[12pt,a4]{article}

\usepackage{graphicx,amsmath,amssymb,amsthm, boxedminipage,xcolor}
\usepackage{epsfig}
\usepackage{graphicx}
\usepackage{subfigure}


\usepackage[lined,boxed]{algorithm2e}
\graphicspath{ {images/} }

\newtheorem{theorem}{Theorem}[section]
\newtheorem{proposition}[theorem]{Proposition}
\newtheorem{lemma}[theorem]{Lemma}
\newtheorem{corollary}[theorem]{Corollary}
\newtheorem{definition}[theorem]{Definition}
\newtheorem{exercise}[theorem]{Exercise}
\newtheorem{exerciseD}[theorem]{*Exercise}
\newtheorem{exerciseDD}[theorem]{**Exercise}

\newenvironment{solution}
  {\renewcommand\qedsymbol{$\blacksquare$}\begin{proof}[Solution]}
  {\end{proof}}

\date{}

\title{
	Mathematical Foundations \\of \\Computer Science\\
	\vspace{3mm}
	{\normalsize CS 499,	Shanghai Jiaotong University,  Dominik Scheder\\}
	{\normalsize Group Name: \textbf{NOIDEA}}
}

\begin{document}
	
	\maketitle
%\begin{quotation}
%  You are welcome to discuss the exercises in the discussion
%  forum. Please take them serious. Doing the exercises is as important
%  than watching the videos.
%
%  I intentionally included very challenging exercises and marked them
%  with one or two ``$*$''. No star means you should be able to solve
%  the exercises without big problems once you have understood
%  the material from the video lecture. One star means it requires 
%  significant additional thinking. Two stars means it is not 
%  unlikely that you will fail to solve them, even once you have understood
%  the material and thought a lot about the exercise. Don't feel bad
%  if you fail. Failure is part of learning.
%
%  This is the first time this course is online. Thus there might be mistakes
%  (typos or more serious conceptual mistakes) in the exercises. I will be 
%  grateful if you point them out to me!
%\end{quotation}


%\setcounter{section}{0}

\section{Broken Chessboard and Jumping With Coins}

\subsection{Tiling a Damaged Checkerboard}

\begin{exercise}
  Re-write the proof in your own way,  using simple English sentences.
\end{exercise}

\begin{proof}
	Your proof ...
\end{proof}

\begin{exercise}
	Another exercise ...
\end{exercise}

\begin{proof}
	Your proof ...
\end{proof}

\section{Exclusion-Inclusion}

\subsection{Sets}

\begin{exercise}
\end{exercise}

\begin{enumerate}
\item \begin{proof}
As is shown in the Venn diagram below, $\left\vert{A}\right\vert + \left\vert{B}\right\vert$ add the common part $\left\vert{A\cap B}\right\vert $ twice. So it should be subtracted once if we want to count $\left\vert{A \cup B}\right\vert $.
\begin{figure}[h]
\centering
\includegraphics[scale=0.1]{Venn}
\caption{Venn Diagram}
\end{figure}
\end{proof}
\item \begin{solution}
$\left\vert{A\cup B\cup C}\right\vert = \left\vert{A}\right\vert + \left\vert{B}\right\vert + \left\vert{C}\right\vert - \left\vert{A\cap B}\right\vert - \left\vert{A\cap C}\right\vert - \left\vert{B \cap C}\right\vert + \left\vert{A\cap B \cap C}\right\vert$
\end{solution}
\item \begin{solution}
$\left\vert{A\cup B\cup C \cup D}\right\vert = \left\vert{A}\right\vert + \left\vert{B}\right\vert + \left\vert{C}\right\vert + \left\vert{D}\right\vert - \left\vert{A\cap B}\right\vert - \left\vert{A\cap C}\right\vert - \left\vert{A \cap D}\right\vert - \left\vert{B \cap C}\right\vert - \left\vert{B \cap D}\right\vert - \left\vert{C \cap D}\right\vert + \left\vert{A\cap B \cap C}\right\vert + \left\vert{A\cap B \cap D}\right\vert + \left\vert{A\cap C \cap D}\right\vert + \left\vert{B\cap C \cap D}\right\vert - \left\vert{A\cap B \cap C\cap D}\right\vert$
\end{solution}
\end{enumerate}

\begin{exercise}
\end{exercise}
	\begin{solution}
$\left\vert{A_1\cup\ldots\cup A_n}\right\vert = \sum\limits_{i=1}^{n}\left\vert{A_i}\right\vert - \sum\limits_{i,j:1\le i<j\le n}\left\vert{A_i\cap A_j}\right\vert + \\
\sum\limits_{i,j,k:1\le i<j<k\le n}\left\vert{A_i\cap A_j \cap A_k}\right\vert - \ldots + (-1)^{n-1}\left\vert{A_1\cap\ldots\cap A_n}\right\vert$
	\end{solution}
\begin{exercise}
\end{exercise}	
\begin{proof}\end{proof}
\begin{enumerate}
		 \item proof using induction on n\\

			 First, let $A_{n,k} = \sum\limits_{1\le i_1<i_2<\ldots\le n}\left\vert{A_{i_1}\cap A_{i_2}\ldots\cap A_{i_k}}\right\vert$, which denotes the sum of all the possible k-wise intersections in $\{A_1,A_2,\ldots,A_n\}$.\\

			 Then the Inclusion-exclusion principle which we want to prove is as follows:\\
			 $\left\vert{A_1\cup\ldots\cup A_n}\right\vert = \sum\limits_{k=1}^{n}(-1)^{k+1}A_{n,k}$\\
		 
		 The theorem holds for $n=1$, obviously.\\
	
		 The theorem holds for $n=2$, as is showed in the $2.1.1$\\

		 For the induction step, we want to show if it holds for $n-1$, then it holds for $n$.\\


		\begin{equation} 
		\begin{split}
		\left\vert{A_1\cup\ldots\cup A_n}\right\vert &= \left\vert{A_1\cup\ldots\cup A_{n-1}}\right\vert + \left\vert{A_n}\right\vert - \left\vert{(A_1\cup\ldots\cup A_{n-1})\cap A_n }\right\vert\\
		&= \sum\limits_{k=1}^{n-1}(-1)^{k+1}A_{n-1,k} + \left\vert{A_n}\right\vert - \left\vert{(A_1\cup\ldots\cup A_{n-1})\cap A_n }\right\vert \\	
		&= \sum\limits_{k=1}^{n-1}(-1)^{k+1}A_{n-1,k} + \left\vert{A_n}\right\vert - \left\vert{(\cup_{i=1}^{i=n-1}(A_i\cap A_n) }\right\vert \\
		\end{split}
		\end{equation}




		 Let $B_i = (A_i\cap A_n)$.\\

		 Similarly, let $B_{n-1,k} = \sum\limits_{1\le i_1<i_2<\ldots\le n-1}\left\vert{B_{i_1}\cap B_{i_2}\ldots\cap B_{i_k}}\right\vert$, which denotes the sum of all the possible k-wise intersections in $\{B_1,B_2,\ldots,B_n-1\}$.\\

		 $(1)$ now becomes\\

		 \begin{equation}
		  \sum\limits_{k=1}^{n-1}(-1)^{k+1}A_{n-1,k} + \left\vert{A_n}\right\vert - \left\vert{(\cup_{i=1}^{i=n-1}B_i }\right\vert 
		 \end{equation}

		 Similarly, it holds:\\
		 \begin{equation}
		 \left\vert{B_1\cup\ldots\cup B_{n-1}}\right\vert = \sum\limits_{k=1}^{n-1}(-1)^{k+1}B_{n-1,k}
		 \end{equation}

		 $(2)$ now becomes\\
		 \begin{equation}
		  \sum\limits_{k=1}^{n-1}(-1)^{k+1}A_{n-1,k} + \left\vert{A_n}\right\vert + (-1)^k \sum\limits_{k=1}^{n-1}B_{n-1,k}
		 \end{equation}	

		 	 




		\item proof not using induction on n
\end{enumerate}

	


\section{third part}
\begin{exercise}
\end{exercise}
Given $A_1,A_2,\ldots,A_n$ and $I\subseteq[n]$, I is not empty.
define $B_I = (\cap_
{i\in I}A_i)\\(\cup_{j\notin I}A_j)$. That is the elements that are in every $A_i,in\in I$
but in no{} other $A_j,j\in [n]\\I$

1.Solve 3.3". Given a B-table, how to determine whether it is feasible.

2.Given a feasible B-table,how to compute A-table.

2.Given an A-table,find a way to compute the B-table.
and then apply 1. 

\section{Feasible Intersection Patterns}

\subsection{}

\begin{exercise}
  Find sets A1; A2; A3; A4 such that all pairwise intersections have size 3 and all three-wise intersections have size 1. Formally,
 1.$|A_i\bigcap A_j| = 3 for all {i,j} \in ^4_2$,
 2.$|A_i\bigcap A_j\bigcap A_k| = 1$ for all ${i,j,k}\in (^{[4]}_{3})$.
\end{exercise}
    \begin{figure}[h]
	\begin{center}
		\includegraphics[width=0.32\linewidth]{Exercise3.1-1.png}
		\caption{picture 1}
		\label{Fig:1}
	\end{center}
	\vspace{-0.5em}
    \end{figure}

\begin{proof}
    from $|A_i\bigcap A_j| = 3$ for all ${i,j} \in ^4_2$.
    We can infer that Domain$\{3,10,12,5\},\\
    \{4,5,10,13\}$,$\{5,6,12,14\},\{5,8,13,14\}$ has each 3 elements.
    From$|A_i\bigcap A_j\bigcap A_k| = 1$ for all $\{i,j,k\}\in (^{[4]}_{3})$
    We can infer that Domain$\{5,10\},\{5,12\},\{5,13\},\{5,14\}$ has each 1 elements.
    {5,8,13,14}have 3 elements, and $\{5,13\} , \{5,14\}$ has each 1 element.\\
    For one thing,Domain$\{8\}\{13\}\{14\}$ is empty and Domain ${5}$ has one element.
    Then it is obvious that 1 in\{5\}, 0 in$\{10\}\{12\}\{13\}\{14\}$, 2 in $\{3\}\{4\}\{6\}\{8\}$. And arbitrary number in \{1\}\{2\}\{7\}\{8\}.
    As it is shown is Figure 3.\\
    \begin{figure}[h]
	\begin{center}
		\includegraphics[width=0.32\linewidth]{Exercise3.1-2.png}
		\caption{}
		\label{Fig:2}
	\end{center}
	\vspace{-0.5em}
    \end{figure}

    For another thing, Domain\{5\}has no element.And then Domain\{10\}\{12\}\{13\}\{14\}\{3\}\{4\}\{6\}\{8\}has either 1 element.

    We can draw a picture .
     \begin{figure}[h]
	\begin{center}
		\includegraphics[width=0.32\linewidth]{Exercise3.1-3.png}
		\caption{}
		\label{Fig:3}
	\end{center}
	\vspace{-0.5em}
    \end{figure}
\end{proof}

\begin{exercise}
\end{exercise}
	In the spirit of the previous questions, let us call a sequence $(a_1,a_2,...,a_n)\in\mathbb{N}_0$ feasible if there are sets $A_1,...,A_n$ such that all k-wise intersections have size $a_k$. That is, $|Ai| = a_{1}$ for all i, $|A_i \bigcap A_j| = a_2$ for all $i \not= j$ and so on. The previous exercise would thus state that $(5, 3, 1, 0)$ is not feasible, but $(6, 3, 1, 0)$ is, as one solution of Exercise 3.1 shows.

\begin{proof}
	Because $|A_1 \bigcap A_2 \bigcap A_3 \bigcap A_4 |=0$.So it is the same as the second situation. From picture 3 in 3.1 we can know that there are at least 6 elements in $A_i$.So $(5,3,1,0)$ is not feasible.
\end{proof}


\begin{exercise}
\end{exercise}
	Suppose I give you a sequence $(a_1,...,a_n)$. Find a way to determine whether such a sequence is feasible or not.



\begin{proof}
Given $A_1,...,A_n I\subseteq{1,...,n}$ define $A_i=\bigcap\limits_{i \in I} A_i$. Given $B_1,...,B_n I\subseteq{1,...,n}$ define $B_i=\bigcap\limits_{i \in I} A_i\bigcup \limits_{j \notin A_i}$.
Use $A_1,A_2,A_3$ as an example,draw the picture below.
 \begin{figure}[h]
	\begin{center}
		\includegraphics[width=0.32\linewidth]{Exercise3.3-1.png}
		\caption{}
		\label{Fig:4}
	\end{center}
	\vspace{-0.5em}
    \end{figure}
    Obviously, iff for every i,$B_i \geq 0$,the B-table is feasible.
    For given $(a_1,...,a_n),A{1}=a_1,A{i,j}(i\not=j)a_2,...,A{i,2,...,n}=a_n$.
    Besides,for all ${i},{i,j}(i \not=j),{i,j...k}$(i,j,...k are different) A{}has the same value.
    So we can infer that $A_i=A\sum_{j=1}^{i-1}A_j$.
    ${A_1,...,A_n}={a_1,...,a_n}$.
    So iff $A_i=A\sum_{j=1}^{i-1}A_j$,the sequence is feasible.
\end{proof}


\end{document}

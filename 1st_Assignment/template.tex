\documentclass[12pt,a4]{article}

\usepackage{graphicx,amsmath,amssymb,amsthm, boxedminipage,xcolor}
\usepackage[lined,boxed]{algorithm2e}

\newtheorem{theorem}{Theorem}[section]
\newtheorem{proposition}[theorem]{Proposition}
\newtheorem{lemma}[theorem]{Lemma}
\newtheorem{corollary}[theorem]{Corollary}
\newtheorem{definition}[theorem]{Definition}
\newtheorem{exercise}[theorem]{Exercise}
\newtheorem{exerciseD}[theorem]{*Exercise}
\newtheorem{exerciseDD}[theorem]{**Exercise}

\date{}

\title{
	Mathematical Foundations \\of \\Computer Science\\
	\vspace{3mm}
	{\normalsize CS 499,	Shanghai Jiaotong University,  Dominik Scheder\\}
	{\normalsize Group Name: \textbf{Your Group Name}}
}

\begin{document}
	
	\maketitle
%\begin{quotation}
%  You are welcome to discuss the exercises in the discussion
%  forum. Please take them serious. Doing the exercises is as important
%  than watching the videos.
%
%  I intentionally included very challenging exercises and marked them
%  with one or two ``$*$''. No star means you should be able to solve
%  the exercises without big problems once you have understood
%  the material from the video lecture. One star means it requires 
%  significant additional thinking. Two stars means it is not 
%  unlikely that you will fail to solve them, even once you have understood
%  the material and thought a lot about the exercise. Don't feel bad
%  if you fail. Failure is part of learning.
%
%  This is the first time this course is online. Thus there might be mistakes
%  (typos or more serious conceptual mistakes) in the exercises. I will be 
%  grateful if you point them out to me!
%\end{quotation}


%\setcounter{section}{0}

\section{Broken Chessboard and Jumping With Coins}

\subsection{Tiling a Damaged Checkerboard}

\begin{exercise}
  Re-write the proof in your own way,  using simple English sentences.
\end{exercise}

\begin{proof}
	Your proof ...
\end{proof}

\begin{exercise}
	Another exercise ...
\end{exercise}

\begin{proof}
	Your proof ...
\end{proof}

\end{document}

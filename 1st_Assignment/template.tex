\documentclass[12pt,a4]{article}

\usepackage{graphicx,amsmath,amssymb,amsthm, boxedminipage,xcolor}


\usepackage[lined,boxed]{algorithm2e}
\graphicspath{ {images/} }

\newtheorem{theorem}{Theorem}[section]
\newtheorem{proposition}[theorem]{Proposition}
\newtheorem{lemma}[theorem]{Lemma}
\newtheorem{corollary}[theorem]{Corollary}
\newtheorem{definition}[theorem]{Definition}
\newtheorem{exercise}[theorem]{Exercise}
\newtheorem{exerciseD}[theorem]{*Exercise}
\newtheorem{exerciseDD}[theorem]{**Exercise}

\newenvironment{solution}
  {\renewcommand\qedsymbol{$\blacksquare$}\begin{proof}[Solution]}
  {\end{proof}}

\date{}

\title{
	Mathematical Foundations \\of \\Computer Science\\
	\vspace{3mm}
	{\normalsize CS 499,	Shanghai Jiaotong University,  Dominik Scheder\\}
	{\normalsize Group Name: \textbf{NOIDEA}}
}

\begin{document}
	
	\maketitle
%\begin{quotation}
%  You are welcome to discuss the exercises in the discussion
%  forum. Please take them serious. Doing the exercises is as important
%  than watching the videos.
%
%  I intentionally included very challenging exercises and marked them
%  with one or two ``$*$''. No star means you should be able to solve
%  the exercises without big problems once you have understood
%  the material from the video lecture. One star means it requires 
%  significant additional thinking. Two stars means it is not 
%  unlikely that you will fail to solve them, even once you have understood
%  the material and thought a lot about the exercise. Don't feel bad
%  if you fail. Failure is part of learning.
%
%  This is the first time this course is online. Thus there might be mistakes
%  (typos or more serious conceptual mistakes) in the exercises. I will be 
%  grateful if you point them out to me!
%\end{quotation}


%\setcounter{section}{0}

\section{Broken Chessboard and Jumping With Coins}

\subsection{Tiling a Damaged Checkerboard}

\begin{exercise}
  Re-write the proof in your own way,  using simple English sentences.
\end{exercise}

\begin{proof}
	Your proof ...
\end{proof}

\begin{exercise}
	Another exercise ...
\end{exercise}

\begin{proof}
	Your proof ...
\end{proof}

\section{Exclusion-Inclusion}

\subsection{Sets}

\begin{exercise}
\end{exercise}

\begin{enumerate}
\item \begin{proof}
As is shown in the Venn diagram below, $\left\vert{A}\right\vert + \left\vert{B}\right\vert$ add the common part $\left\vert{A\cap B}\right\vert $ twice. So it should be subtracted once if we want to count $\left\vert{A \cup B}\right\vert $.
\begin{figure}[h]
\centering
\includegraphics[scale=0.1]{Venn}
\caption{Venn Diagram}
\end{figure}
\end{proof}
\item \begin{solution}
$\left\vert{A\cup B\cup C}\right\vert = \left\vert{A}\right\vert + \left\vert{B}\right\vert + \left\vert{C}\right\vert - \left\vert{A\cap B}\right\vert - \left\vert{A\cap C}\right\vert - \left\vert{B \cap C}\right\vert + \left\vert{A\cap B \cap C}\right\vert$
\end{solution}
\item \begin{solution}
$\left\vert{A\cup B\cup C \cup D}\right\vert = \left\vert{A}\right\vert + \left\vert{B}\right\vert + \left\vert{C}\right\vert + \left\vert{D}\right\vert - \left\vert{A\cap B}\right\vert - \left\vert{A\cap C}\right\vert - \left\vert{A \cap D}\right\vert - \left\vert{B \cap C}\right\vert - \left\vert{B \cap D}\right\vert - \left\vert{C \cap D}\right\vert + \left\vert{A\cap B \cap C}\right\vert + \left\vert{A\cap B \cap D}\right\vert + \left\vert{A\cap C \cap D}\right\vert + \left\vert{B\cap C \cap D}\right\vert - \left\vert{A\cap B \cap C\cap D}\right\vert$
\end{solution}
\end{enumerate}

\begin{exercise}
\end{exercise}
	\begin{solution}
$\left\vert{A_1\cup\ldots\cup A_n}\right\vert = \sum\limits_{i=1}^{n}\left\vert{A_i}\right\vert - \sum\limits_{i,j:1\le i<j\le n}\left\vert{A_i\cap A_j}\right\vert + \\
\sum\limits_{i,j,k:1\le i<j<k\le n}\left\vert{A_i\cap A_j \cap A_k}\right\vert - \ldots + (-1)^{n-1}\left\vert{A_1\cap\ldots\cap A_n}\right\vert$
	\end{solution}
\begin{exercise}
\end{exercise}
	\begin{proof}
	

	\end{proof}


\begin{enumerate}
\item testddd
\item testddd
\end{enumerate}

\end{document}

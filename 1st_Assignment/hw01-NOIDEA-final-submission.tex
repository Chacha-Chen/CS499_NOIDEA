\documentclass[12pt,a4]{article}

\usepackage{graphicx,amsmath,amssymb,amsthm, boxedminipage,xcolor}
\usepackage[lined,boxed]{algorithm2e}
\usepackage{epsfig}
\usepackage{float}
\usepackage{subfig}
\usepackage{graphicx}
\usepackage{amsthm}

\newtheoremstyle{exercise}{}{}{}{}{\bf}{.}{ }{}
\graphicspath{ {images/} }

\theoremstyle{exercise}
\newtheorem{theorem}{Theorem}[section]
\newtheorem{proposition}[theorem]{Proposition}
\newtheorem{lemma}[theorem]{Lemma}
\newtheorem{corollary}[theorem]{Corollary}
\newtheorem{definition}[theorem]{Definition}
\newtheorem{exercise}[theorem]{Exercise}
\newtheorem{exerciseD}[theorem]{*Exercise}
\newtheorem{exerciseDD}[theorem]{**Exercise}

\newenvironment{solution}
  {\renewcommand\qedsymbol{$\blacksquare$}\begin{proof}[Solution]}
  {\end{proof}}

\date{}

\title{
	Mathematical Foundations \\of \\Computer Science\\
	\vspace{3mm}
	{\normalsize CS 499,	Shanghai Jiaotong University,  Dominik Scheder\\}
	{\normalsize Group Name: \textbf{NOIDEA}}
}
 \begin{document}
\maketitle

\section{Broken Chessboard and Jumping With Coins}

\subsection{Tiling a Damaged Checkerboard}

\begin{exercise}
    Re-write the proof in your own way,  using simple English sentences.
\end{exercise}

\begin{proof}
	Let's color the board black and beige, just like a typical chessboard.
    In this case, there are 32 black and 30 beige squares.

    \begin{figure}[h]
        \small
        \centering
        \includegraphics[width=5cm]{slightlydamaged.png}
        \label{fig:slightlyDamagedBoard}
        \caption{The Chessboard}
    \end{figure}

    Since a domino stone placed in the board will cover both a black square and a beige square,
    it's obviously that we can't fully tile the board with $1 \times 2$ domino stones.

\end{proof}

\begin{exercise}
    Look at the seriously damaged $8 \times 8$ checkerboard.
    For convenience I already colored it black and white (or rather black and beige):

    Try to tile it with domino stones and you will fail.
    However, since there are $24$ black and $24$ beige squares, the simple argument from the lecture will fail.

    Prove that the above board cannot be tiled. Try to find a short and simple argument!
\end{exercise}

\begin{proof}
	To prove the claim, let's start by focusing on the ``corridor" in the chessboard.
    \begin{figure}[h]
        \small
        \centering
        \includegraphics[width=5cm]{corridor.png}
        \label{fig:corridor}
        \caption{The ``Corridor"}
    \end{figure}

    There are 7 ways to tile the ``corridor" with domino stones:

    \begin{figure}[H]
        \centering
        \subfloat[]{\includegraphics[width=3cm]{corridor_a.png}}
        \subfloat[]{\includegraphics[width=3cm]{corridor_b.png}}
        \subfloat[]{\includegraphics[width=3cm]{corridor_c.png}}
        \subfloat[]{\includegraphics[width=3cm]{corridor_d.png}} \\
        \subfloat[]{\includegraphics[width=3cm]{corridor_e.png}}
        \subfloat[]{\includegraphics[width=3cm]{corridor_f.png}}
        \subfloat[]{\includegraphics[width=3cm]{corridor_g.png}}
	\label{fig:corridorTile}
	\caption{Tiling the Corridor}
    \end{figure}

    In the case (a) and (b), each of the two remaining parts of the chessboard contains 21 square.
    Since a domino stone tiles 2 sqaures, the  (a) and the case (b) are ruled out.

    How about (c) ,(d), (e), (f) and (g)? We consider the possible remaining parts in these cases:

    \begin{figure}[H]
        \centering
        \subfloat{\includegraphics[width=2cm]{left_a.png}}
        \subfloat{\includegraphics[width=2cm]{left_b.png}}
        \subfloat{\includegraphics[width=2cm]{left_c.png}}
        \subfloat{\includegraphics[width=2cm]{left_d.png}}
	\label{fig:leftPart}
	\caption{The Remaining Parts}
    \end{figure}

    Becasue the number of black squares and beige squares is not equal, all of the possible remaining parts cannot be tiled.

\end{proof}




\subsection{Jumping with Coins}

\begin{exercise}
    You jump around with two coins.
    Show that you cannot increase the distance between the two coins.
\end{exercise}

\begin{proof}
    Clearly, the distance between two coins is fixed in one jump.

    \begin{figure}[H]
        \small
        \centering
        \includegraphics[width=5cm]{twocoins.png}
        \label{fig:twoCoins}
        \caption{Two Coins}
    \end{figure}

\end{proof}


\begin{exercise}
    You jump around with three coins.
    Show that you cannot start with an equilateral triangle and end up with a bigger equilateral triangle.
    Give a simple proof!
\end{exercise}

\begin{proof}
    During a jump, either the side or the height of the triangle will not change, thus the area stays the same.

    \begin{figure}[H]
        \small
        \centering
        \includegraphics[width=5cm]{threecoins.png}
        \label{fig:threeCoins}
        \caption{Three Coins}
    \end{figure}

\end{proof}

You jump around with four coins which in the beginning form a square of side length $1$.

\begin{exercise}
    Show that you cannot form a square of side length $2$.
\end{exercise}

\begin{proof}
     We use the coordinate to express the position of coins. The initial coordinate will be
    $$
    (odd, odd), (odd, even), (even, even), (even, odd).
    $$
    If a coin at $(a, b)$ jump over the coin at $(c, d)$, the new position $(e, f)$ will be
    $$
    e = 2c - a, f = 2d - b
    $$

  Both $2c$ and $2d$ are even, so the parity of $e, f$ will be the same as $a, b$. It means that the parity of coordinate of a coin will not change during a jump.

  If a square of side lenth $2$ is formed, the parity of the coordinate will be different with the initial one, which is impossible.
\end{proof}

\begin{exercise}
    Show that you cannot achieve a position in which two coins are at the same position.
\end{exercise}

\begin{proof}
    We have proved that the parity of coordinate of a coin will not change during a jump. Hence two coins with different initial parity of the coordinate can not be at the same position. 
    
    And we know the all the four coins have different initial parity, so there won't be any coins at the same position.
\end{proof}

\begin{exercise}
    Show that you cannot form a larger square.
\end{exercise}

\begin{proof}
   If we can make the four coins form a larger square with side length $n > 1$ in a few jumps. Then if the initial side length is $1/n$ we can use the same steps of jump to form a square of side length $1$.
   
    It means that with the converse steps, We can form a smaller square with side length $1/n < 1$ with initial side length $1$, which requiring that the coordinates of coins are not integer.
    
     Obviously, it's impossible.
\end{proof}

\section{Exclusion-Inclusion}

\subsection{Sets}

\begin{exercise}
\end{exercise}

\begin{enumerate}
\item \begin{proof}
As is shown in the Venn diagram below, $\left\vert{A}\right\vert + \left\vert{B}\right\vert$ add the common part $\left\vert{A\cap B}\right\vert $ twice. So it should be subtracted once if we want to count $\left\vert{A \cup B}\right\vert $.
\begin{figure}[h]
\centering
\includegraphics[scale=0.1]{Venn}
\caption{Venn Diagram}
\end{figure}
\end{proof}
\item \begin{solution}
$\left\vert{A\cup B\cup C}\right\vert = \left\vert{A}\right\vert + \left\vert{B}\right\vert + \left\vert{C}\right\vert - \left\vert{A\cap B}\right\vert - \left\vert{A\cap C}\right\vert - \left\vert{B \cap C}\right\vert + \left\vert{A\cap B \cap C}\right\vert$
\end{solution}
\item \begin{solution}
$\left\vert{A\cup B\cup C \cup D}\right\vert = \left\vert{A}\right\vert + \left\vert{B}\right\vert + \left\vert{C}\right\vert + \left\vert{D}\right\vert - \left\vert{A\cap B}\right\vert - \left\vert{A\cap C}\right\vert - \left\vert{A \cap D}\right\vert - \left\vert{B \cap C}\right\vert - \left\vert{B \cap D}\right\vert - \left\vert{C \cap D}\right\vert + \left\vert{A\cap B \cap C}\right\vert + \left\vert{A\cap B \cap D}\right\vert + \left\vert{A\cap C \cap D}\right\vert + \left\vert{B\cap C \cap D}\right\vert - \left\vert{A\cap B \cap C\cap D}\right\vert$
\end{solution}
\end{enumerate}

\begin{exercise}
\end{exercise}
	\begin{solution}
$\left\vert{A_1\cup\ldots\cup A_n}\right\vert = \sum\limits_{i=1}^{n}\left\vert{A_i}\right\vert - \sum\limits_{i,j:1\le i<j\le n}\left\vert{A_i\cap A_j}\right\vert + \\
\sum\limits_{i,j,k:1\le i<j<k\le n}\left\vert{A_i\cap A_j \cap A_k}\right\vert - \ldots + (-1)^{n-1}\left\vert{A_1\cap\ldots\cap A_n}\right\vert$
	\end{solution}
\begin{exercise}
\end{exercise}	
\begin{proof}\end{proof}
\begin{enumerate}
		 \item proof using induction on n\\

			 First, let $A_{n,k} = \sum\limits_{1\le i_1<i_2<\ldots\le n}\left\vert{A_{i_1}\cap A_{i_2}\ldots\cap A_{i_k}}\right\vert$, which denotes the sum of all the possible k-wise intersections in $\{A_1,A_2,\ldots,A_n\}$.\\

			 Then the Inclusion-exclusion principle which we want to prove is as follows:\\
			 $\left\vert{A_1\cup\ldots\cup A_n}\right\vert = \sum\limits_{k=1}^{n}(-1)^{k+1}A_{n,k}$\\
		
		 The theorem holds for $n=1$, obviously.\\
	
		 The theorem holds for $n=2$, as is showed in the $2.1.1$\\

		 For the induction step, we want to show if it holds for $n-1$, then it holds for $n$.\\


		\begin{equation}
		\begin{split}
		\left\vert{A_1\cup\ldots\cup A_n}\right\vert &= \left\vert{A_1\cup\ldots\cup A_{n-1}}\right\vert + \left\vert{A_n}\right\vert - \left\vert{(A_1\cup\ldots\cup A_{n-1})\cap A_n }\right\vert\\
		&= \sum\limits_{k=1}^{n-1}(-1)^{k+1}A_{n-1,k} + \left\vert{A_n}\right\vert - \left\vert{(A_1\cup\ldots\cup A_{n-1})\cap A_n }\right\vert \\	
		&= \sum\limits_{k=1}^{n-1}(-1)^{k+1}A_{n-1,k} + \left\vert{A_n}\right\vert - \left\vert{(\cup_{i=1}^{i=n-1}(A_i\cap A_n) }\right\vert \\
		\end{split}
		\end{equation}




		 Let $B_i = (A_i\cap A_n)$.\\

		 Similarly, let $B_{n-1,k} = \sum\limits_{1\le i_1<i_2<\ldots\le n-1}\left\vert{B_{i_1}\cap B_{i_2}\ldots\cap B_{i_k}}\right\vert$, which denotes the sum of all the possible k-wise intersections in $\{B_1,B_2,\ldots,B_n-1\}$.\\

		 $(1)$ now becomes\\

		 \begin{equation}
		  \sum\limits_{k=1}^{n-1}(-1)^{k+1}A_{n-1,k} + \left\vert{A_n}\right\vert - \left\vert{(\cup_{i=1}^{i=n-1}B_i }\right\vert
		 \end{equation}

		 Similarly, it holds:\\
		 \begin{equation}
		 \left\vert{B_1\cup\ldots\cup B_{n-1}}\right\vert = \sum\limits_{k=1}^{n-1}(-1)^{k+1}B_{n-1,k}
		 \end{equation}

		 $(2)$ now becomes\\
		 \begin{equation}
		  \sum\limits_{k=1}^{n-1}(-1)^{k+1}A_{n-1,k} + \left\vert{A_n}\right\vert +  \sum\limits_{k=1}^{n-1}(-1)^kB_{n-1,k}
		 \end{equation}	
		 In addition,
	    \begin{equation}
		\left\vert{A_n}\right\vert = (-1)^{1+1}\left\vert{A_n}\right\vert
  		\end{equation}	
  		Thus,
  		\begin{equation}
  		\left\vert{A_n}\right\vert+\sum\limits_{k=1}^{n-1}(-1)^kB_{n-1,k} = \sum\limits_{k=1}^{n}(-1)^{k+1}A_{n,k} - \sum\limits_{k=1}^{n-1}(-1)^{k+1}A_{n-1,k}
  		\end{equation}	

		 Then equation(4) finally becomes: $\sum\limits_{k=1}^{n}(-1)^{k+1}A_{n,k}$

		 	




		\item proof not using induction on n\\

    First, let $A = \left\vert{A_1\cup\ldots\cup A_{n-1}}\right\vert $. \\
    Function $P_S(x)$ defined as if set $S$ includes element $x$, then $P_S(x) = 1$, else $P_S(x) = 0$.\\

    $(1)$ If $P_A(x) = 1$, there must exist an $i$ that $P_{A_i}(x) = 1$. In this way:\\
    $(P_A(x) - P_{A_1}(x))(P_A(x) - P_{A_2}(x))\ldots(P_A(x) - P_{A_n}(x)) = 0$\\

    $(2)$ According to the properities of set, $P_{A_i}(x)P_{A_j}(x) = P_{A_i \cap A_j}(x)$.\\

    $(3)$ Let $P_{n,k}$ denotes $P_{A_{i_1}\cap A_{i_2}\ldots\cap A_{i_k}}(1\le i_1<i_2<\ldots\le n)$.\\
    Then decompose the first equation, we can have:\\
    $P_A(x) = \sum\limits_{k=1}^{n}P_{n,k}$\\
    which can be demonstrated as:\\
     $\left\vert{A_1\cup\ldots\cup A_n}\right\vert = \sum\limits_{k=1}^{n}(-1)^{k+1}A_{n,k}$\\
\end{enumerate}

\section{Feasible Intersection Patterns}

\subsection{}

\begin{exercise}
\end{exercise}
  Find sets $A_1, A_2, A_3, A_4$ such that all pairwise intersections have size 3 and all three-wise intersections have size 1. \\Formally,
 1.$|A_i\bigcap A_j| = 3$ for all ${i,j} \in (^{[4]}_{2})$,
 2.$|A_i\bigcap A_j\bigcap A_k| = 1$ for all $\{i,j,k\}\in (^{[4]}_{3})$.

    
    
    $A_1=\{1,2,3,5,6,9\}$;
    
    $A_2=\{1,2,3,4,7,8\}$;
    
    $A_3=\{1,5,7,8,9,10\}$;
    
    $A_4=\{2,4,5,6,8,10\}$;
    
\begin{exercise}
\end{exercise}
	In the spirit of the previous questions, let us call a sequence $(a_1,a_2,...,a_n)\in\mathbb{N}_0$ feasible if there are sets $A_1,...,A_n$ such that all k-wise intersections have size $a_k$. That is, $|Ai| = a_{1}$ for all i, $|A_i \bigcap A_j| = a_2$ for all $i \not= j$ and so on. The previous exercise would thus state that $(5, 3, 1, 0)$ is not feasible, but $(6, 3, 1, 0)$ is, as one solution of Exercise 3.1 shows.

\begin{proof}
\end{proof}

Assume that there exist such sets A,B,C,D to which $(5,3,1,0)$ is feasible.

$|A|=|B|=|C|=|D|=5$;

Since $|A\bigcap B|=3$,$|A\bigcap B\bigcap C|=1$,

A and B have 3 same elements. A, B and C have 1.

There are 2 elements in $A\bigcap B$ that are not in C;

Since $|A\bigcap C|=3$, there are 2 elements in A and C but not in B.

There are 2 elements in B and C but not in A.

So, now we have 1 elements in A, B, C, 2 in A,C but not B, 2 in B,C but not A.

Then, set D has the same requirement with C.

Similarly,We have 1 elements in A, B, D, 2 in A,D but not B, 2 in B,D but not A.

A has only $5$ elements. So dose B.

Then there is a contradiction. D will have at least $5$ same elements with C.

$|C\bigcap D|=5\not = 3$.

\begin{exercise}
\end{exercise}
	Given a sequence $(a_1,...,a_n)$, Find a way to determine whether such a sequence is feasible or not.

\begin{proof}
\end{proof}

 \textbf{Definition}: 
 
     Given $A_1,A_2,A_3,...,A_n$, 
     
     define 
     \begin{center}
      $A_{\{I\}}=\bigcap \limits_{i \in I} A_i $. 
     \end{center} 
     In other words, 
     \begin{center}
      $A_{\{i,j,k,...\}}= A_i \bigcap A_j \bigcap A_k \bigcap ...$
      
      $a_{|I|}=|A_{\{I\}}|.$
     
      $B_{\{I\}}=\bigcap \limits_{i \in I} A_i / \bigcup \limits_{j\notin I }{A_j}$.
     \end{center}
     
     In other words,
     \begin{center}
      $B_{\{I\}}$ is the number of elements in $\bigcap \limits_{i \in I} A_i$ but not in $A_j,j\notin I$.
     
      $b_{|I|}=|B_{\{I\}}|.$
     \end{center}
     
     Because every set $B_i$ is not divided by existing boundaries, we can put any number of elements in set $B_i$.
     
     \textbf{a.Whether a given B-table is feasible:}

     Thus, $(b_1,b_2,...,b_n)$ is feasible for all situations as long as $b_i$ are nonnegative integers.

     Obviously, $(a_1,a_2,...,a_n)$ has something to do with $(b_1,b_2,...,b_n)$.We have
     \begin{center}
     \textbf{b.Given a feasible B-table, how to compute A-table:}

     $A_{\{I\}}=\bigcup \limits_{J\ge I}B_{\{J\}}$;

     Thus, $a_{|I|}=\sum \limits_{J\ge I}b_{|J|}=\sum \limits_{j=i}^{j=n} (_{j-i}^{n-i}) b_{|J|}$
     
     \textbf{c.Given a A-table, how to compute B-table:}

     Then,$b_i=\sum \limits_{k=0}^{n-i} (-1)^k (_{k}^{n-i} )a_{i+k}$
     
     \end{center}

     \textbf{d.Then apply a:}
     
     For $(a_1,a_2,...,a_n)$, replace the $a_i$ into the formula.
     
     If all the results are nonnegative integers.The sequence is feasible.
     

\end{document}

\end{document}

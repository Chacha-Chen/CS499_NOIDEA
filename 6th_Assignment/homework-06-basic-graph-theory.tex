\documentclass[12pt,a4]{article}




\usepackage{graphicx,amsmath,amssymb,amsthm, boxedminipage,xcolor}

\usepackage{algorithm}
\usepackage{algpseudocode}


\newtheorem{theorem}{Theorem}[section]
\newtheorem{proposition}[theorem]{Proposition}
\newtheorem{lemma}[theorem]{Lemma}
\newtheorem{corollary}[theorem]{Corollary}
\newtheorem{definition}[theorem]{Definition}

\newtheorem*{theorem*}{Theorem}
\newtheorem*{lemma*}{Lemma}
\newtheorem*{proposition*}{Proposition}


\newtheorem{exercise}[theorem]{Exercise}
\newtheorem{exerciseD}[theorem]{*Exercise}
\newtheorem{exerciseDD}[theorem]{**Exercise}

\let\oldexercise\exercise
\renewcommand{\exercise}{\oldexercise\normalfont}


\newcommand{\E}{\mathbb{E}}
\newcommand{\scalar}[2]{\ensuremath{\langle #1, #2\rangle}}
\newcommand{\floor}[1]{\left\lfloor #1 \right\rfloor}
\newcommand{\ceil}[1]{\left\lceil #1 \right\rceil}
\newcommand{\norm}[1]{\|#1\|}
\newcommand{\pfrac}[2]{\left(\frac{#1}{#2}\right)}
\newcommand{\nth}[1]{#1^{\textsuperscript{th}}}
\newcommand{\core}{\textnormal{core}}





\newcommand{\poly}{\textnormal{poly}}
\newcommand{\quasipol}{\textnormal{quasipol}}
\newcommand{\ssubexp}{\textnormal{stronglySubExp}}
\newcommand{\wsubexp}{\textnormal{weaklySubExp}}
\newcommand{\simplyexp}{\textnormal{E}}
\newcommand{\expo}{\textnormal{Exp}}



\newcommand{\N}{\mathbb{N}}
\newcommand{\nn}{\mathbb{N}_0^n}
\newcommand{\R}{\mathbb{R}}
\newcommand{\Z}{\mathbb{Z}}


\definecolor{darkgreen}{rgb}{0,0.6,0}


\date{}

\title{
  Mathematical Foundations \\of \\Computer Science\\
  \vspace{3mm}
{\normalsize CS 499,	Shanghai Jiaotong University,  Dominik Scheder}
}

\begin{document}

\maketitle

%\begin{quotation}
%  You are welcome to discuss the exercises in the discussion
%  forum. Please take them serious. Doing the exercises is as important
%  than watching the videos.
%
%  I intentionally included very challenging exercises and marked them
%  with one or two ``$*$''. No star means you should be able to solve
%  the exercises without big problems once you have understood
%  the material from the video lecture. One star means it requires 
%  significant additional thinking. Two stars means it is not 
%  unlikely that you will fail to solve them, even once you have understood
%  the material and thought a lot about the exercise. Don't feel bad
%  if you fail. Failure is part of learning.
%
%  This is the first time this course is online. Thus there might be mistakes
%  (typos or more serious conceptual mistakes) in the exercises. I will be 
%  grateful if you point them out to me!
%\end{quotation}





\setcounter{section}{5}


\section{Graph Theory Basics}



\begin{itemize}
 \item Homework assignment published on Monday, 2018-04-02.
 \item Submit first solutions and questions by Sunday, 2018-04-08, 12:00, by
 email to dominik.scheder@gmail.com and to the TAs.
 \item You will receive feedback by Wednesday, 2018-04-11.
 \item Submit final solution by Sunday, 2018-04-15 to me and the TAs.
\end{itemize}


Let $G = (V, E)$ and $H = (V', E')$ be two graphs. A {\em graph isomorphism} from $G$ to $H$
 is a bijective function $f: V \rightarrow V'$ such that
 for all $u,v \in V$ it holds that $\{u,v\} \in E$ if and only if $\{f(u),f(v)\} \in E'$. 
 If such a function exists, we write $G \cong H$ and say that $G$ and $H$ are {\em isomorphic}.
 In other words, $G$ and $H$ being isomorphic means that they 
 are identical up to the names of its vertices.
 
Obviously, every graph $G$ is isomorphic to itself, because the identity function $f(u) = u$ is an
isomorphism. However, there might be several isomorphisms $f$ from $G$ to $G$ itself.
We call such an isomorphism from $G$ to itself an {\em automorphism} of $G$.

\begin{exercise}
For each of the graphs below, compute the number of automorphisms it has.
\begin{center}
\includegraphics[width=\textwidth]{figures/graphs-number-of-automorphisms.pdf}
\end{center}
Justify your answer!
\end{exercise}

\begin{solution}
  \quad \newline
  \begin{enumerate}
    \item The number of automorphisms is $2$. We denote the vertices $v_1, v_2, v_3, v_4$ from the left side to the right side, which itself is an automorphisms. The other automorphism is to denote the vertices from the right to the left.

    \item The number of automorphisms is $2\times 6 = 12$. For each vertex, we denote it $v_1$ and number the remaining vertex clockwise or anticlockwise in order, all of which are automorphisms.

    \item The number of automorphisms is $8\times 6 = 48$. For each vertex, we denote it $v_0$, and its adjacent three vertices $v_1,v_2,v_3$, which has $3!=6$ choices. Each selection can form an automorphism, thus the total number is $48$.
  \end{enumerate}
\end{solution}


Consider the $n$-dimensional Hamming cube $H_n$. This is the graph with vertex
set $\{0,1\}^n$, and two vertices $x,y \in \{0,1\}^n$ are connected by an edge if 
they differ in exactly one edge. For example, the right-most graph in the figure above
is $H_3$.

\begin{exercise}
   Show that $H_n$ has exactly $2^n \cdot n!$ automorphisms.
   Be careful: it is easy to construct $2^n \cdot n!$ different automorphisms. It is 
   more difficult to show that there are no automorphisms other than those.
\end{exercise}

\begin{proof}
  \quad \newline
  \begin{itemize}
    \item There are at least $2^n\cdot n!$ automorphisms. For each vertex we denote it $v_0$ , and denote its adjacent vertices $v_1, v_2, \ldots, v_n$, which has $n!$ choice. All $2^n\cdot n!$ will at least form one automorphism.

    \item There are no more than $2^n\cdot n!$ automorphisms. If there are more than $2^n\cdot n!$ automorphisms, then after we choose $v_0, v_1, v_2, \ldots, v_n$, we can still find more than one automorphism when numbering $v_{n+1}, v_{n+2}, \ldots, v_{2^n-1}$.

    Consider the n-dimensional Hamming cube as a n-dimensional linear space with each vertex having the coordinates $(x_1, x_2, \ldots, x_n)$ ($x_i = 1$ or $0$ ) However, after choosing $v_0, v_1, \ldots, v_n$, equivalently, we have defined the origin and basis of the space, all vertices' coordinates are fixed, there is no alternative automorphism.
  \end{itemize}
\end{proof}

A graph $G$ is called {\em asymmetric} if the identity function $f(u) = u$ is the 
only automorphism of $G$. That is, if $G$ has exactly one automorphism.

\begin{exercise}
   Give an example of an asymmetric graph on six vertices.
\end{exercise}

\begin{solution}
  \begin{figure}[h]
    \centering
    \includegraphics[width=4cm]{figures/6.3.png}
  \end{figure}
\end{solution}

\begin{exercise}
  Find an asymmetric tree.
\end{exercise}

\begin{solution}
  The is no asymmetric tree on six vertices.
  \begin{proof}
    A tree containing only $2$ leaves is a path, which has $2$ automorphisms. Thus the tree must containing at least $3$ leaves $v_1, v_2, v_3$ and the leaves have different adjacent vertices $v_1', v_2', v_3'$. Otherwise switching the number of $2$ leaves connecting the same adjacent vertex will form a new automorphism.

    There already exist $6$ vertices. But if there are only 6 vertice to form a tree, $v_1', v_2', v_3'$ must be connected. But this will form automorphisms by switch the name of vertices groups $(v_1,v_1'), (v_2,v_2'), (v_3,v_3')$.

    Therefore, an asymmetric tree cannot be found on less than $7$ vertices.
  \end{proof}
\end{solution}

For a graph $G = (V,E)$, let $\bar{G} := \left(V, {V \choose 2} \setminus E \right)$ denote
its {\em complement graph}.
\begin{center}
\includegraphics[width=0.5\textwidth]{figures/graph-complement.pdf}
\end{center}
We call a graph {\em self-complementary} if $G \cong \bar{G}$. The above graph
is not self-complementary. Here is an example of a self-complementary graph:
\begin{center}
\includegraphics[width=0.5\textwidth]{figures/pentagon-pentagram.pdf}
\end{center}

\begin{exercise}
   Show that there is no self-complementary graph on $999$ vertices.
\end{exercise}

\begin{proof}
  A complete graph on $999$ vertices has $999\times 998 /2 = 498501$ edges, which cannot be divide by $2$, that is, the edges cannot be divide into two parts with the same number of edges.
\end{proof}

\begin{exercise}
 Characterize the natural numbers $n$ for which there is a self-complementary
 graph $G$ on $n$ vertices. That is, state and prove a theorem of the form
 ``There is a self-complementary graph  on $n$ vertices if and only if 
 $n$ \texttt{ <put some simple criterion here>}.''
\end{exercise}

\begin{theorem*}
    There is a self-complementary graph on n vertices if and only if $n\mod 4 = 0$ or $n\mod 4 = 1$.
  \end{theorem*}
  \begin{proof}
    \quad \newline
    \begin{itemize}
      \item If $n\mod 4 \neq 0$ and $n\mod 4 \neq 1$, the number of edges of the complete graph $n(n-1)/2/2$ is not an integer, namely the edges of the complete graph cannot be divide into two parts with the same number of edges. Hence there isn't any self-complementary graph on $n$ vertices.

      \item If $n\mod 4 = 0$, that is $n = 4k$, where $k$ is a positive integer. we denote an graph $G_k$ containing $k$ vertices whose complement is $G_k^C$. Now we form the following graph $G$ on $n$ vertices

      \begin{figure}[h]
        \centering
        \includegraphics[width=5cm]{figures/graph4k.png}
      \end{figure}
      The graph $G$ contains four induced subgraphs $G_k, G_k, G_k^C, G_k^C$. Between some subgraphs there are edges between each vertex of two subgraphs, whose total number is $k^2$.

      The complementary graph $G^C$ is as follows. Obviously, $G^C$ and $G$ are isomorphic.

      \begin{figure}[h]
        \centering
        \includegraphics[width=5cm]{figures/graphc4k.png}
      \end{figure}

      \item If $n\mod 4 = 1$, that is $n = 4k +1$, where $k$ is a positive integer.

      Based on the graph we formed when $n\mod 4 = 1$, we add one vertex to form a new graph $G'$ on $4k+1$ vertices and its complement ${G^C}'$.

      \begin{figure}[h]
        \centering
        \includegraphics[width=5cm]{figures/graph4k1.png}
      \end{figure}

      \begin{figure}[h]
        \centering
        \includegraphics[width=6cm]{figures/graphc4k1.png}
      \end{figure}

      Similarly, $G'$ and ${G^C}'$ are isomorphic.
    \end{itemize}


  \end{proof}





\end{document}
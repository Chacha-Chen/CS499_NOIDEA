\documentclass[12pt,a4]{article}




\usepackage{graphicx,amsmath,amssymb,amsthm, boxedminipage,xcolor}

\usepackage{algorithm}
\usepackage{algpseudocode}


\newtheorem{theorem}{Theorem}[section]
\newtheorem{proposition}[theorem]{Proposition}
\newtheorem{lemma}[theorem]{Lemma}
\newtheorem{corollary}[theorem]{Corollary}
\newtheorem{definition}[theorem]{Definition}

\newtheorem*{theorem*}{Theorem}
\newtheorem*{lemma*}{Lemma}
\newtheorem*{proposition*}{Proposition}


\newtheorem{exercise}[theorem]{Exercise}
\newtheorem{exerciseD}[theorem]{*Exercise}
\newtheorem{exerciseDD}[theorem]{**Exercise}

\let\oldexercise\exercise
\renewcommand{\exercise}{\oldexercise\normalfont}


\newcommand{\E}{\mathbb{E}}
\newcommand{\scalar}[2]{\ensuremath{\langle #1, #2\rangle}}
\newcommand{\floor}[1]{\left\lfloor #1 \right\rfloor}
\newcommand{\ceil}[1]{\left\lceil #1 \right\rceil}
\newcommand{\norm}[1]{\|#1\|}
\newcommand{\pfrac}[2]{\left(\frac{#1}{#2}\right)}
\newcommand{\nth}[1]{#1^{\textsuperscript{th}}}
\newcommand{\core}{\textnormal{core}}





\newcommand{\poly}{\textnormal{poly}}
\newcommand{\quasipol}{\textnormal{quasipol}}
\newcommand{\ssubexp}{\textnormal{stronglySubExp}}
\newcommand{\wsubexp}{\textnormal{weaklySubExp}}
\newcommand{\simplyexp}{\textnormal{E}}
\newcommand{\expo}{\textnormal{Exp}}



\newcommand{\N}{\mathbb{N}}
\newcommand{\nn}{\mathbb{N}_0^n}
\newcommand{\R}{\mathbb{R}}
\newcommand{\Z}{\mathbb{Z}}


\definecolor{darkgreen}{rgb}{0,0.6,0}


\date{}

\title{
  Mathematical Foundations \\of \\Computer Science\\
  \vspace{3mm}
{\normalsize CS 499,	Shanghai Jiaotong University,  Dominik Scheder}
}

\begin{document}

\maketitle

%\begin{quotation}
%  You are welcome to discuss the exercises in the discussion
%  forum. Please take them serious. Doing the exercises is as important
%  than watching the videos.
%
%  I intentionally included very challenging exercises and marked them
%  with one or two ``$*$''. No star means you should be able to solve
%  the exercises without big problems once you have understood
%  the material from the video lecture. One star means it requires 
%  significant additional thinking. Two stars means it is not 
%  unlikely that you will fail to solve them, even once you have understood
%  the material and thought a lot about the exercise. Don't feel bad
%  if you fail. Failure is part of learning.
%
%  This is the first time this course is online. Thus there might be mistakes
%  (typos or more serious conceptual mistakes) in the exercises. I will be 
%  grateful if you point them out to me!
%\end{quotation}





\setcounter{section}{8}

\section{Hamilton Cycles, Hamilton Paths, and Non-isomorphic Trees}



\begin{itemize}
 \item Homework assignment published on Monday 2018-04-23
 \item Submit your first solution by Sunday, 2018-04-29, 18:00, by email
 \item Submit your final solution by Sunday, 2018-05-06.
\end{itemize}

\subsection{Regular Orientations of a Regular Graph}

We call a graph {\em $d$-regular} if every vertex has degree $d$. 
A directed graph is $(d,d)$-regular if every vertex has $d$ incoming and $d$
outgoing edges.

\begin{exercise}
   Show that in every $4$-regular graph, you can orient the edges such that every vertex
   has two incoming and two outgoing edges, i.e., such that the resulting
   digraph is $(2,2)$-regular. See the picture below for an illustration.
\end{exercise}
\begin{center}
\includegraphics[width=\textwidth]{figures/four-regular-oriented.pdf}
\end{center}

\begin{proof}

A graph $G$ has an Eulerian Walk if and only if $G$ is connected and evety vertex in $G$ has even degree.\\
Every $4$-regular graph satisfies above criterions therefore containing Eulerian Walk. We orient edges along the Eulerian walk. Since Eulerian walk can cover all the edges only once ending at start vertex, every vertex in $4$-regular graph has two incoming and two outcoming edges.\\

\end{proof}
\subsection{Hamilton Cycles and Ore's Theorem}

Consider $K_n$, the complete graph on $n$ vertices. For $n \geq 3$, this obviously has a Hamilton cycle.
How many edges do you have to delete from $K_n$ to destroy all Hamilton cycles? That is, what is the 
smallest set $S$ such that $\left( V, {V \choose 2} \setminus S\right)$ has no Hamilton cycle? Let $s_n$ denote
the size of this set (this depends on $n$, thus the notation $s_n$). For example, $s_2 = 0$ since $K_2$
has no Hamilton path to begin with; $s_3 = 1$ since removing one edge from $K_3$ results in a graph without
a Hamilton cycle.

\begin{exercise}
 Find a closed formula for $s_n$ and prove it! \textbf{Hint.} One part will be easy. For the other part,
 use Ore's Theorem.
\end{exercise}


\begin{solution}
  The formular is $s_n = n - 2$ when $n > 2$.

  First, if we remove $n-2$ edges of a vertex $v$ in $K_n$ and get graph $G$, then $v$ has degree one. Obviously $G$ doesn't contain Hamilton cycle. We have $s_n \le n - 2$.

  Second, if we remove $n-3$ deges of a vertex $v$ in $K_n$ and get graph $G_1$, then $v$ has degree $2$ and vertexs adjacent to $v$ have degree $n-1$. Degree of all vertexs can be demonstrated as $(2,n-1,n-1,n-2 \dots,n-2)$. According to Ore's theorem, $G_1$ contain Hamilton Cycle.

  To estimate all cases when $s_n = n - 3$, we add $k$ edges adjacent to $v$ and remove $k$ other edges in $G_1$ instead. We can find that the degree of all vertexs in $G_1$ still meet Ore's theorem. In other words, we can't destory all Hamilton Cycle when $s_n = n - 3$. Thus, $s_n > n -3$.

  Since $s_n > n-3$ and $s_n \le n-2$, we have $s_n = n-2$.

\end{solution}

A {\em tournament} is a directed graph in which, for each pair $u,v \in V$, exactly one of the 
directed edges $(u,v)$ and $(v,u)$ is in the graph. Imagine a sports tournament in which every participant
plays against every other exactly once. Draw an arc from $u$ to $v$ if $u$ beat $v$ in this tournament.

\begin{center}
\includegraphics[width=0.5\textwidth]{figures/tournament-five.pdf}\\
{\small A tournament on five vertices.}
\end{center}

\begin{exercise}
   Show that every tournament has a {\em directed Hamilton path}, i.e., a sequence
   $u_1,u_2,\dots,u_n$ such that $(u_i, u_{i+1}) \in E$ for all $i = 1,\dots, n-1$. See the picture below.

\begin{center}
\includegraphics[width=0.5\textwidth]{figures/tournament-five-path.pdf}\\
{\small The same tournament wiht a Hamilton path.}
\end{center}
  You probably won't be able to use the proof of Ore's Theorem directly, but you can use the proof idea.
\end{exercise}
\begin{proof}


\begin{enumerate}
	\item  We can use an induction to prove it.
	\item Base case: when there are 2 vertices, clearly there is a Hamilton path.See the picture bellow.

	\item Now a graph with k vertices has a Hamilton path. Add a vertex $u$ to it.

	Along the path, numbered the vertices as their explored order as $(v_1,v_2...v_k)$.

	\item If the edge between $v_k$ and $u$ is $(v_k,u)$ ,add $u$ as the tail of the path.Then we get $(v_1,...v_k,u)$.
	
	\item Else turn to ${v_k-1}$, if there is $(v_{k-1},u)$, clearly we can insert u into the path:$(v_1,v_2...v_k,u,v_{k+1},v_k)$.

	\item Else turn to the vertex before continously until there is $(v_i,u)$, Clearly we can insert u into the path:$(v_1,v_2...v_i,u,v_{i+1},...,v_k)$.

	\item Else all edges are from from $u$ to $v_i$, add $u$ as the head of the path.Then we get $(v_1,...v_k,u)$ or $(u,v_1,...v_k)$.

	\item Above all, every tournament has a {\em directed Hamilton path}.
\end{enumerate}
\end{proof}
\subsection{Isomorphism Classes of Trees}

In the lecture (and in the videos) we have seen that the number of trees on 
vertex set $V = \{1,2,\dots,n\}$ is $n^{n-2}$. This however ignores isomorphisms.
For example, there are $3^{3-2} = 3$ trees on vertex set $\{1,2,3\}$, but all those 
trees look alike (are isomorphic). On $\{1,2,3,4\}$, there are 16 trees, but 
there are only two isomorphism classes: the path and the star. For five vertices, there are 125 trees
but only three isomorphism classes: the path, the star, and the ``T-shape'' (see video on
counting the number of trees). For $n=6$ we get the path, the Y-shape, the Euro symbol, the Star Wars fighter, the Scandinavian cross, and the star, so six isomorphism classes (but a total of 1296 trees).

\begin{exercise}
  List of isomorphism classes on seven vertices. That is, draw trees $T_1,\dots,T_m$ on seven vertices such that 
  no two of them are isomorphic but every tree on seven vertices is isomorphic to one of them. 
  How many do you get?
\end{exercise}




\begin{tabular}{r|c|c|c|c|c|c|c}
$n$ & 1 & 2 & 3 & 4 & 5 & 6 & 7 \\ \hline
number of isomorphism classes & 
          1 & 1 & 1 & 2 & 3 & 6 & ? 
\end{tabular}
\vspace{5mm}

\begin{solution}


We get 9 different trees.
\begin{center}
\includegraphics[width=\textwidth]{figures/Figure9-4-1.pdf}\\
{\small $T_1,T_2...T_9$}
\end{center}
\end{solution}

Alright, so let's denote by $t_n$ the number of isomorphism classes of trees on $n$ vertices.
That is, $t_n$ is the largest number $m$ such that we can find trees $T_1,\dots,T_m$ on 
$n$ vertices such that no two of them are isomorphic. We would like to have an exact and
 explicit formula for $t_n$, but that is probably too much to ask for.
Instead, let us try to understand $t_n$ approximately and asymptotically.

\begin{exercise}
  Show that $t_n \leq 4^n$. Hint: Consider the video on the isomorphism problem on trees.
  It defines a way to encode a tree as a $0/1$-sequence.
\end{exercise}
\begin{proof}

Encode a tree as 0/1-sequence in the way below.

If there is a vertex ,we write 1 to explore it and then write 0 to return.

Take the following tree as an example.
\begin{center}
\includegraphics[width=0.58\textwidth]{figures/Figure9-5-1.pdf}\\
%{\small $T_1,T_2...T_9$}
\end{center}

Then if we have n vertices, we have a sequence of 2n bits.

There are $2^{2n}=4^n$ trees.

However, in a tree if the leftchild tree of a node is the same as the rightchild tree of the same node, they are isomorphisms.BUt they have different 0/1-sequences.

Thus $t_n\le 4^n$
\end{proof}
\begin{exercise}
  Show that $t_n \geq \frac{e^n}{\poly(n)}$, where $\poly(n)$ is some polynomial in $n$.
  Hint: There are $n^{n-2}$ trees on $V = [n]$. We group them together in ``buckets''
  of isomorphic trees. How large can a bucket be? Answer this and then use Stirling's approximation
  for $n!$.
\end{exercise}

\begin{proof}
	For a tree with $n$ vertices, it is less than $n!$ naming methods for the same structure.
	\begin{align}
	t_n\ge \frac{n^{n-2}}{n!}=\frac{n^n}{n^2\times n!}
	\end{align}
	From Stirling's approximation, 
	\begin{align}
	n!  \approx\sqrt{2\pi n}(\frac{n}{e})^n
	\end{align}
	Then 
	\begin{align}
	t_n\ge \frac{e^n}{n^2\sqrt{2\pi n}}.
	\end{align}
	Let \begin{align}poly(n)=n^2\sqrt{2\pi n}.\end{align}

	We have $t_n \geq \frac{e^n}{\poly(n)}$, where $\poly(n)$ is some polynomial in $n$.
\end{proof}
\begin{exerciseDD}
   Try to improve those bounds. That is, find some $a < 4$ such that $t_n \in O(a^n)$ or 
   some $b > e$ such that $t_n \in \Omega(b^n)$. Any improvement will be kind of interesting.
   Aim for simple proofs!
\end{exerciseDD}

\textbf{Remark.} The ``true'' rate of growth is known by a result of George P\'olya but apparently 
it is quite difficult (I write ``apparently'' because I have never studied this work).


\end{document}
\documentclass[12pt,a4]{article}





\usepackage{graphicx,amsmath,amssymb,amsthm, boxedminipage,xcolor}

\usepackage{algorithm}
\usepackage{algpseudocode}


\newtheorem{theorem}{Theorem}[section]
\newtheorem{proposition}[theorem]{Proposition}
\newtheorem{lemma}[theorem]{Lemma}
\newtheorem{corollary}[theorem]{Corollary}
\newtheorem{definition}[theorem]{Definition}

\newtheorem*{theorem*}{Theorem}
\newtheorem*{lemma*}{Lemma}
\newtheorem*{proposition*}{Proposition}


\newtheorem{exercise}[theorem]{Exercise}
\newtheorem{exerciseD}[theorem]{*Exercise}
\newtheorem{exerciseDD}[theorem]{**Exercise}

\let\oldexercise\exercise
\renewcommand{\exercise}{\oldexercise\normalfont}


\newcommand{\E}{\mathbb{E}}
\newcommand{\scalar}[2]{\ensuremath{\langle #1, #2\rangle}}
\newcommand{\floor}[1]{\left\lfloor #1 \right\rfloor}
\newcommand{\ceil}[1]{\left\lceil #1 \right\rceil}
\newcommand{\norm}[1]{\|#1\|}
\newcommand{\pfrac}[2]{\left(\frac{#1}{#2}\right)}
\newcommand{\nth}[1]{#1^{\textsuperscript{th}}}
\newcommand{\core}{\textnormal{core}}





\newcommand{\poly}{\textnormal{poly}}
\newcommand{\quasipol}{\textnormal{quasipol}}
\newcommand{\ssubexp}{\textnormal{stronglySubExp}}
\newcommand{\wsubexp}{\textnormal{weaklySubExp}}
\newcommand{\simplyexp}{\textnormal{E}}
\newcommand{\expo}{\textnormal{Exp}}



\newcommand{\N}{\mathbb{N}}
\newcommand{\nn}{\mathbb{N}_0^n}
\newcommand{\R}{\mathbb{R}}
\newcommand{\Z}{\mathbb{Z}}


\definecolor{darkgreen}{rgb}{0,0.6,0}


\date{}

\title{
  Mathematical Foundations \\of \\Computer Science\\
  \vspace{3mm}
{\normalsize CS 499,	Shanghai Jiaotong University,  Dominik Scheder}
}

\begin{document}

\maketitle

%\begin{quotation}
%  You are welcome to discuss the exercises in the discussion
%  forum. Please take them serious. Doing the exercises is as important
%  than watching the videos.
%
%  I intentionally included very challenging exercises and marked them
%  with one or two ``$*$''. No star means you should be able to solve
%  the exercises without big problems once you have understood
%  the material from the video lecture. One star means it requires 
%  significant additional thinking. Two stars means it is not 
%  unlikely that you will fail to solve them, even once you have understood
%  the material and thought a lot about the exercise. Don't feel bad
%  if you fail. Failure is part of learning.
%
%  This is the first time this course is online. Thus there might be mistakes
%  (typos or more serious conceptual mistakes) in the exercises. I will be 
%  grateful if you point them out to me!
%\end{quotation}


\setcounter{section}{3}



\begin{itemize}
 \item Monday, 2018-03-19, homework handed out
 \item Sunday, 2018-03-25, 12:00: submit questions and first submissions. You'll get feedback
 until Wednesday.
 \item Sunday, 2018-03-29 (Wednesday), 18:00: submit your review of the other group's first submission.
 \item 2018-04-01: submit final solution.
\end{itemize}



\section{Pascal's Triangle Modulo 2}




Here are two early tables of the binomial coefficient:
\begin{center}
  \includegraphics[width=\textwidth]{figures/two-pascal-triangles.pdf}
\end{center}
Here is my version of ``Pascal's triangle'', indicating that rows are indexes by $n$ and ``columns''
by $k$:
\begin{center}
  \includegraphics[width=0.8\textwidth]{figures/pascals-triangle.pdf}
\end{center}

\subsection{Lucas Theorem: ${n \choose k} \mod 2$}


Something interesting happens when we take the triangle modulo $2$, that is, we replace
even numbers by $0$ and odd numbers by $1$:
\begin{center}
  \includegraphics[width=0.8\textwidth]{figures/sierpinski-small.pdf}
\end{center}
If we draw a black dot for every $1$ and look at a larger section of this triangle, we get
the following pattern, known as the Sierpinski triangle:
\begin{center}
  \includegraphics[width=0.8\textwidth]{figures/sierpinski-large.pdf}
\end{center}

Note the amazing recursive structure. This suggests we should be able to 
compute ${n \choose k} \mod 2$ without actually computing ${n \choose k}$,
by somehow employing this structure. In fact, here is a cool result
by \'Edouard Lucas, which we state here in a simpler, more special version:

The set $\N_0$ comes equipped with a partial ordering $\preceq$, in which
$x \preceq y$ if for every $i$, the $\nth{i}$ least significant bit of $x$ is 
at most that of $y$. Put in a simpler way, we write $x$ and $y$ as bit strings
in binary. If their length differ, we put a bunch of $0$'s in front of the smaller number
to make both strings of equal length $d$. Then we simply compare those strings
using the usual partial ordering $\preceq$ on $\{0,1\}^d$. For example,
$3 \preceq 7$ since $011 \preceq 111$, and $5 \preceq 23$ since $00101 \preceq 10111$,
but $7 \not \preceq 8$ since $0111\not \preceq 1000$.


\begin{theorem}
  Let $n, k\in \N_0$. Then ${n \choose k}$ is odd if $k \preceq n$ and even otherwise.
  \label{theorem-lucas-2}
\end{theorem}

Note that this theorem lets us compute ${n \choose k} \mod 2$ quickly for numbers $n,k$
having millions of digits, whereas no computer on Earth has the memory to evaluate 
the formula

\begin{align*}
{n \choose k} & = \frac{n \cdot (n-1) \cdot (n-2) \cdot \dots \cdot (n-k+2) \cdot (n-k+1)}
{k \cdot (k-1) \cdot (k-2) \cdot \dots \cdot 2 \cdot 1}
\end{align*}
for values that large. Let me now walk you through a proof of this theorem. 

\begin{definition}
  For a natural number $n \in \N$, let $|n|_1$ be the number of $1$'s in the binary
  representation of $n$. For example, $|1|_1 = |2|_1 = |4|_1 = 1$ but
  $|3|_1 = 2$ and $|7|_1 = 3$. 
\end{definition}

\begin{definition}
 For a natural number $a \in \N$ define $f(a)$ as the number of times the factor
 $2$ appears in $a$. Formally,
 \begin{align*}
 f(a) := \max \{ k \ | \ 2^k \textnormal{ divides } a\} \ .
\end{align*}
For example, $f(24) = 3$ since $8$ divides $24$ but $16$ does not.
\end{definition}

\begin{exercise}
   Find a closed formula for $f(n!)$ in terms of $n$ and $|n|_1$.
\end{exercise}

\begin{solution} 
\end{solution}
First, for divisor 2, there are $[\frac{x}{2}]$ numbers have the divisor 2. 

For  divisor 4, there are $[\frac{x}{4}]$ numbers have the divisor 4.

So for divisor $2^i$, there are $[frac{x}{2^i}]$ numbers have the divisor $2^i$.

Then,for $x!$ it have divisors between 1 and $x$. Thus, we have the formula for $f(x!)=\sum\limits_{i=1}^\infty {x\over {2^i}} $.

Next, let's assume $\delta(x)=[x \ mode \ 2]$. So, $x=[{x \over 2}] +[{x \over 2}] + {\delta(x)}$.

Obviously in this way we have $$x=\sum_{i=1}^\infty {x\over {2^i}} +\sum_{i=1}^\infty {\delta ({x\over {2^i}})} = f(x!)+\sum_{i=1}^\infty {\delta ({x\over {2^i}})}$$.

Surprisingly, $\sum\limits_{i=1}^\infty {\delta ({x\over {2^i}})}$ is exactly $|n|_1$, 
for every time we divide $x$ by 2 it equals moving the binary code to right by one digit.

In conclusion, $f(n!)=n-|n|_1$.


\begin{exercise}
   Find a closed formula for $f\left({n \choose k}\right)$ in terms of $n, k, |n|_1$, and so on.
\end{exercise}

\textbf{Solution:}\\

\[{n \choose k}={{n!}\over {({n-k})! k!}}\]
\[f\left({n \choose k}\right)={f(n!)+f((n-k)!)-f(k!)}={{n-|n|_1- (k-|k|_1)-(n-k-|n-k|_1)}}\]

\begin{exercise}
 Prove Theorem~\ref{theorem-lucas-2}. With our new notation, prove that
 $f\left( {n \choose k}\right)$ is $0$ if $k \preceq n$ and at least $1$ if 
 $k \not \preceq n$.
\end{exercise}

\begin{solution}
\end{solution}
 With the conclusion of 4.5, what we have to prove is that $|k|_1+|n-k|_1-|n|_1$.

Case1: In $k+(n-k)$, every digit does not produce a carry bit. In this situation, obviously $|k|_1+|n-k|_1-|n|_1$ is correct.

Case2: In $k+(n-k)$, there exists some digit that will produce a carry bit, in binary system which must be $1+1$. 

Then $n$ ,the result of the addition, in this digit should be 0, which is contradicted to the condition that  $k \preceq n$.

(If there are more than 1 carry bit, which can make some digit in $n$ still be 1, we can simply trace back to the lowest digit that produces the carry bit and then the proof is universal.)


\subsection{Almost Empty Rows}

One feature of the Sierpinski triangle is that some rows are almost empty. For example, row 64
has a black dot at the very left and the very right, and only white space in between. This is because
\begin{theorem}
  Let $d \in \N_0$ and $0 < k < 2^d$. Then ${2^d \choose k}$ is even.
  \label{theorem-binomial-even}
\end{theorem}
Although this theorem follows easily from Lucas' Theorem, I want you to think about an alternative proof.
Intuitively, if some number is even, then one suspects it can be proved by ``pairing things up'' perfectly.
After all, if you can prove that in a set $S$, every element can be ``married'' to another element,
you have partitioned $S$ into couples and thus $|S|$ must be even. So let's see whether
there is a proof of Theorem~\ref{theorem-binomial-even} along these lines.
This is also
valuable because it lets you practice with notions of sets and functions. \\

Consider
the set $\{0,1\}^d$. You can view this as the set of all binary strings of length $d$.
This set has size $2^d=n$.
For $1 \leq i \leq d$ and $x \in \{0,1\}^d$ let $f_i(x)$ be $x$ with the $\nth{i}$ position
flipped. For example, $f_3(11011) = 11111$.
\begin{exercise}
 Show that $f_i$ is an involution without a fixed point.
 That is, $f(f(x)) = x$ and $f(x) \ne x$ for all $x \in \{0,1\}^d$.
\end{exercise}


\begin{proof}
If x has n bits, write down x as $x=a_1a_2a_3...a_n$.

$\forall$ $j=1,2,3,...n$, $a_j=0 or 1$.

If $a_j$ is flipped, record it as $\bar a_j$.

Then $\forall$ $i=1,2,3,...n$, $f_i(x)=a_1a_2a_3...\bar a_i...a_n$

Then $f(f_i(x))=a_1a_2a_3...\bar{\bar a_i}...a_n=a_1a_2a_3...a_i...a_n=x$.

If there is $x_k $ making $f(x)=x$,.

Then for an i, $x=a_1a_2a_3...a_n=a_1a_2a_3...\bar a_i...a_n$.

$a_i=\bar a_i$.

It is impossible.

Above all,$f_i$ is an involution without a fixed point.


\end{proof} 

Let $S \subseteq \{0,1\}^d$. We define  $f_i(S)$ as the set arising from applying
$f_i$ to every element of $S$. Formally,
\begin{align*}
f_i(S) := \{ f_i(x) \ | \ x \in S \} \ .
\end{align*}
Given a set $S \subseteq \{0,1\}^d$, we call an index $i \in [n]$ {\em active} for $S$
if $f_i(S) \ne S$. 
\begin{exercise}
Let $d=3$ and $S = \{000, 100\}$. Which of the indices $1,2,3$ are active?
\end{exercise}

\begin{solution}
$f_1(S)=\{100,000\}=S$

$f_2(S)=\{010,110\}=S$

$f_1(S)=\{001,101\}=S$

2,3 are active.
\end{solution}

\begin{exercise}
 Let $S \subseteq \{0,1\}^d$. Show that if $S \ne \emptyset$ and $S \ne \{0,1\}^d$ then
 $S$ has at least one active index.
\end{exercise}

\begin{proof}
Soppose S has no active index and $S \ne \emptyset$.
%When $d=k$, record S as $S_k$

Then $\forall i\in\{1,2,3...d\}$, $f_i(S)=S$.

%We use induction to verify it.

$S=\{s_{1},s_{2}...s_n\}$

$s_j=a_{1}a_{2}...a_{d}$

%\item Base case: $i=1$.$a_1\in S_1.$

So $f_i(s_j)= a_1a_2...\bar a_ja_n\in S$.

So take a sequence s from S, $\forall i$, when the $i^th$ position is flipped,the new $s'\in S$.

Then through change every bit of s and its child sequences, we can get every sequences of $\{0,1\}^d$.

$s=\{0,1\}^d$.

So if $S \ne \emptyset$ and $S \ne \{0,1\}^d$ then
 $S$ has at least one active index.

%\item Induction:$d=k$,$S_d = \{0,1\}^d$.

%When $d=k+1$, record $S_{k+1}=\{s_{k1}a_1,s_{k2}a_2\}$

\end{proof}

Given $S \subseteq \{0,1\}^d$, define $f(S)$ as follows: if $S = \emptyset$ or $S = \{0,1\}^d$
define $f(S) = S$. Otherwise, let $f(S) := f_i(S)$ where $i$ is the smallest active
index of $S$ (which exists by the previous exercise).

\begin{exercise}
 Show that $f$ is an involution. That is, $f(f(S)) = S$. Furthermore, show that
 the only fixed points of $f$ are $\emptyset$ and $\{0,1\}^d$.
\end{exercise} 

\begin{proof}
From Ex 4.8, we know $\forall i=\{1,2...d\}$,$\forall x\in S$, $f(f(x))=x$.

Then $\forall s\in S$, $f(f(s))=s$.

So $f(f(S))=S$.

If f has fixed point, namely $f(S)=S$.

From Ex 4.10, we know that it is impossible that $f(S)=S$ unless $S = \emptyset$ or $S = \{0,1\}^d$.

Above all,$f(f(S)) = S$. And the only fixed points of $f$ are $\emptyset$ and $\{0,1\}^d$.
\end{proof}

\begin{exercise}
   Let $\mathcal{S} = { \{0,1\}^d \choose k }$. This is a set of sets, and each set $S \in \mathcal{S}$
   consists of exactly $k$ strings from $\{0,1\}^d$. Prove the following statements: 
   \begin{enumerate}
   \item $f$ is a bijection from $\mathcal{S}$ to $\mathcal{S}$.
   \item For $1 \leq k \leq 2^d-1$, this bijection is an involution
   without fixed points.
   \item $|\mathcal{S}|$ is even for $1 \leq k \leq 2^d-1$.
   \end{enumerate}
\end{exercise}

\begin{proof}
\end{proof}

\begin{enumerate}
	\item For a sequence $\in S$, $f(s)$is a certain sequence.

	So for $S=\{s_1,s_2...,s_k\}$,$f(S)=\{f(s_1),f(s_2),...,f(s_k)\}$is a certain set.

	Then f is surjective.

	If for $a,b\in \mathcal{S}$,$a\ne b$,$f(a)=f(b)=A$.

	From exercise above we know,$f(f(a))=a$ and $f(f(b))=b$,namely $f(A)=a$,$f(A)=b$

	which is contradictive with f is surjective.

	So f is bijective.

	\item In 1 we have proved $f(f(S))=S$,it is an involution.

	If f has fixed points, namely f(S)=S.

	From Ex 4.11, the only fixed points of $f$ are $\emptyset$ and $\{0,1\}^d$.

	However $k\in [1,2^d-1]$, f has no fixed points.

	For $1 \leq k \leq 2^d-1$, this bijection is an involution without fixed points.
	
 	\item  	Since $\mathcal{S}$ is bijective, it has a number of couples $(S_i,S_j)$ in which $f(S_i)=S_j$.

 	So $|\mathcal{S}|$ is even.

 	

\end{enumerate}

\begin{exercise}
  Complete the proof of Theorem~\ref{theorem-binomial-even}.
\end{exercise}

\begin{proof}
\end{proof}

$|\mathcal{S}|={ {\{0,1\}^d} \choose k }$,then $|\mathcal{S}|=$ $2^d \choose k$.

$|\mathcal{S}|$ is even.

Then $2^d \choose k$ is even.

\begin{exerciseD}
  Generalize the above ``combinatorial'' proof to show the following theorem:
  \begin{theorem}
 Let $n = p^d$ where $p$ is a prime number. Then $p$ divides
 ${n \choose k}$ unless $k=0$ or $k=n$. 
 \end{theorem}
\end{exerciseD}

\begin{proof}
\end{proof}

\begin{enumerate}
	\item 
	Let $S\subseteq {0,1,2,3...p-1}^d$.

	For $1\le i \le d$ and $x \in \{0,1\}^d$ let $g_i(x)$ be x with the $i^th$ position replaced with x'. 0 is replaced by 1, 1 by 2...$p-2$ by $p-1$,$p-1$ by 0.

	\item
	Let $S \subseteq \{0,1...p-1\}^d$. We define  $g_i(S)$ as the set arising from applying $g_i$ to every element of $S$. Formally,

	\begin{align*}
	g_i(S) := \{ g_i(x) \ | \ x \in S \} \ .
	\end{align*}

	Given a set $S \subseteq \{0,1\}^d$, we call an index $i \in [n]$ {\em active} for $S$
	if $g_i(S) \ne S$. 

	Given a set $S \subseteq \{0,1...p-1\}^d$, we call an index $i \in [n]$ {\em active} for $S$ if $g_i(S) \ne S$. 

	\item
	Similar to Ex 4.12, let Q = $\{0,1...p-1\}^d \choose k$.

	Every p sequences are in a group, in which $g(s_1)=s_2$,$g_(s_2)=s_3$...$g_(s_{p-1})=s_p$,$g(s_p)=s_1$.

	\item
	Then $|Q|$ can be divided by p.

	Then $p$ divides
	 ${n \choose k}$ unless $k=0$ or $k=n$. 
\end{enumerate}

\end{document}
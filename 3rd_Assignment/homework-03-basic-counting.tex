\documentclass[12pt,a4]{article}







\usepackage{graphicx,amsmath,amssymb,amsthm, boxedminipage,xcolor}

\usepackage{algorithm}
\usepackage{algpseudocode}


\newtheorem{theorem}{Theorem}[section]
\newtheorem{proposition}[theorem]{Proposition}
\newtheorem{lemma}[theorem]{Lemma}
\newtheorem{corollary}[theorem]{Corollary}
\newtheorem{definition}[theorem]{Definition}

\newtheorem*{theorem*}{Theorem}
\newtheorem*{lemma*}{Lemma}
\newtheorem*{proposition*}{Proposition}


\newtheorem{exercise}[theorem]{Exercise}
\newtheorem{exerciseD}[theorem]{*Exercise}
\newtheorem{exerciseDD}[theorem]{**Exercise}

\let\oldexercise\exercise
\renewcommand{\exercise}{\oldexercise\normalfont}


\newcommand{\E}{\mathbb{E}}
\newcommand{\scalar}[2]{\ensuremath{\langle #1, #2\rangle}}
\newcommand{\floor}[1]{\left\lfloor #1 \right\rfloor}
\newcommand{\ceil}[1]{\left\lceil #1 \right\rceil}
\newcommand{\norm}[1]{\|#1\|}
\newcommand{\pfrac}[2]{\left(\frac{#1}{#2}\right)}
\newcommand{\nth}[1]{#1^{\textsuperscript{th}}}
\newcommand{\core}{\textnormal{core}}





\newcommand{\poly}{\textnormal{poly}}
\newcommand{\quasipol}{\textnormal{quasipol}}
\newcommand{\ssubexp}{\textnormal{stronglySubExp}}
\newcommand{\wsubexp}{\textnormal{weaklySubExp}}
\newcommand{\simplyexp}{\textnormal{E}}
\newcommand{\expo}{\textnormal{Exp}}



\newcommand{\N}{\mathbb{N}}
\newcommand{\nn}{\mathbb{N}_0^n}
\newcommand{\R}{\mathbb{R}}
\newcommand{\Z}{\mathbb{Z}}


\definecolor{darkgreen}{rgb}{0,0.6,0}


\date{}

\title{
  Mathematical Foundations \\of \\Computer Science\\
  \vspace{3mm}
{\normalsize CS 499,	Shanghai Jiaotong University,  Dominik Scheder}
}

\begin{document}

\maketitle

%\begin{quotation}
%  You are welcome to discuss the exercises in the discussion
%  forum. Please take them serious. Doing the exercises is as important
%  than watching the videos.
%
%  I intentionally included very challenging exercises and marked them
%  with one or two ``$*$''. No star means you should be able to solve
%  the exercises without big problems once you have understood
%  the material from the video lecture. One star means it requires 
%  significant additional thinking. Two stars means it is not 
%  unlikely that you will fail to solve them, even once you have understood
%  the material and thought a lot about the exercise. Don't feel bad
%  if you fail. Failure is part of learning.
%
%  This is the first time this course is online. Thus there might be mistakes
%  (typos or more serious conceptual mistakes) in the exercises. I will be 
%  grateful if you point them out to me!
%\end{quotation}



\newenvironment{solution}
  {\renewcommand\qedsymbol{$\blacksquare$}\begin{proof}[Solution]}
  {\end{proof}}
\setcounter{section}{2}


\begin{itemize}
 \item Homework assignment published on Tuesday, 2018-03-13
 \item Submit questions and first solutions by Sunday, 2018-03-18, 12:00 by email to dominik.scheder@gmail.com  and the TAs.
 \item You will receive feedback by Wednesday, 2018-03-21
 \item Revise your solution and submit your final solution by Sunday, 2018-03-25 by  email to dominik.scheder@gmail.com and the TAs.
\end{itemize}



\section{Basic Counting}

A function $[m] \rightarrow [n]$ is {\em monotone} if $f(1) \leq f(2) \leq \dots \leq f(m)$.
It is  {\em strictly monotone} if $f(1) < f(2) < \dots < f(m)$.

\begin{exercise}
   Find and justify a closed formula for the number of strictly
   monotone functions from $[m]$ to $[n]$.
\end{exercise}

\begin{proof}
  ${n \choose m}$.

  Since a strictly monotone function is needed which means that $f(1),...,f(m)$ are not equal with each other, $m$ different elements should be chosen from the set $[n]$, and each is assigned to $f(i), i=1,2,...,m$,  $s.t. f(1) < f(2) < \dots < f(m)$. Hence, the number of strictly monotone function is ${n \choose m}$.
\end{proof}

\begin{exercise}
   Find and justify a closed formula for the number of monotone functions from $[m]$ to $[n]$.
\end{exercise}

\begin{proof}
  First of all, a sequence of $m$ elements in a non-decreasing order(repeated elements are allowed)should be chosen from the set $[n]$ to form a monotone function.
  Without loss of generality, assume the set $[n]$ as $\{n_1,\dots,n_n\}$.\\ 
  Let $num_i$ be the number of $n_i$ chosen, $i=1,2,...,n$. 

  $num_1+num_2+...+num_n = m$ , $s.t.$ $0\le num_i\le m$ and $num_i\in N$

Transform the problem into:

  $num_1+1+num_2+1+...+num_n+1 = m+n$ , $s.t.$ $1\le (num_i+1)\le (m+1)$ and $num_i\in N^+$

  Hence, then the number of all possible solutions of the equation above is ${m+n-1 \choose n-1} = {m+n-1 \choose m}$, which is also the number of monotone functions.


\end{proof}


\textbf{Remark.} By ``closed'' I mean something using expressions like $\times$, $+$,
${n \choose k}$, $n!$, but not $\sum$ or $\prod$.
For example, ${n \choose k^2}$ is a closed formula but
$\sum_{k=0}^n {n \choose k}$ is not.


\begin{exercise}
  Prove that $\sum_{k=0}^n {n \choose k}^2 = {2n \choose n}$ for every $n \geq 0$ by finding a combinatorial interpretation.
\end{exercise}

\proof We divide the set of $2n$ elements into $2$ sets of $n$ elements. For each $0 \leq k \leq n$, we pick out $k$ elements from one set, exclude $k$ elements in the other set and combine them, we will get $n$ elements, which is equivalent to selecting $n$ elements from $2n$ elements.

Therefore, $\sum_{k=0}^n {n \choose k}^2 = {2n \choose n}$ for every $n \geq 0$.


\begin{exercise}[From the textbook]
   Find a closed formula for $\sum_{k=m}^n {k \choose m}{n \choose k}$ and prove it combinatorially, i.e., by giving an
   interpretation.
\end{exercise}

\begin{solution}
  $\sum_{k=m}^n {k \choose m}{n \choose k} = {n \choose m}\cdot 2^{n-m}$.
  \proof The formula means for each $m \leq k \leq n$, we select $k$ elements from $n$ elements, then select $m$ elements from these $k$ elements, which is equivalent to first choose $n$ elements from $m$ elements, then decide whether to pick out the remaining elements that is ${n \choose m}\cdot 2^{n-m}$.
\end{solution}

\begin{exercise}
   Let $B_n$ be the number of partitions of the set $[n]$ (this is the same as the number
   of equivalence relations on $[n]$). This is called the Bell number, thus we
   denote it $B_n$.
   Prove that the following recursive formula
   for $B_n$ is correct:
   \begin{align*}
     B_0 & = 1 \\
     B_{n+1} & = \sum_{k=0}^n {n \choose k} B_k  \ .
   \end{align*}
\end{exercise}

\proof $B_0 = 1$ is correct, undoubtedly.

Denote the elements in the set $\lbrace x_1, x_2, \ldots, x_n, \ldots \rbrace$.

Assume that for $B_0, B_1, \ldots, B_n$, the formula is right. Then we add a element $x_{n+1}$ to the set. How many options do we have?

For each $0 \leq k \leq n$, we select $n - k$ elements from the former $n$ elements and combine them with the element $n_{k+1}$,which contains ${n \choose n-k} = {n \choose k}$ ways, then partition the remaining $k$ elements. Therefore we have $ B_{n+1} = \sum_{k=0}^n {n \choose k} B_k$.

\begin{exercise}
  Let $P_n$ be the number of ways to write the natural number $n$ as a sum $
  a_1 + a_2 + \cdots + a_k$ such that
  $1 \leq a_1 \leq a_2 \leq \dots \leq a_k$. For example, $3$ can be written
  as $3$, $2 + 1$, and $1 + 1 + 1$, so $P_3 = 3$.
   Find a recursive formula for $P_n$.\\

   \textbf{Remark.} The formula might not be as simple as the above for $B_n$. Be creative!
   Start by writing a simple recursive program that computes $P_n$.
\end{exercise}

\begin{solution}
Let $G_i(n)$ be the number of ways to wirte the natural number $n$ as a sum of $i$ numbers following the rules $1 \leq a_1 \leq a_2 \leq \dots \leq a_i$. Speacially we define $G_i(k) = 0$ where $i > k$.\\
Then we have $P_n = G_1(n) + G_2(n) + \dots + G_n(n)$.\\
Consider wirting $n$ as a sum of $i$ numbers $a_1, a_2, \dots a_i$ with $1 \leq a_1 \leq a_2 \leq \dots \leq a_i$, and $a_1, a_2 \dots a_k$ are $1$s, $a_{k+1} \dots a_i$ are larger than $1$:\\
Substract these numbers by 1, then we get $0, 0 \dots 0, a_{k+1}^*, \dots a_i^*$.\\
For any $k$ with $1 \leq k \leq i$, the sum of $a_{k+1}^*, \dots a_i^*$ is $(n-i)$. Then the number of ways to wirte $i$ numbers which includes $k$ $1$s is equal to $G_{i-k}(n-i)$.\\
In this way, $G_i(n) = \sum_{k=0}^{i-1}{G_{i-k}(n-i)}$ and  $P_n = \sum_{i=1}^{n-1}{\sum_{k=0}^{i-1}{G_{i-k}(n-i)}} + G_n(n) = \sum_{i=1}^{n-1}{\sum_{k=0}^{i-1}{G_{i-k}(n-i)}} + 1$\\
Therefore $P_n - P_{n-1} = \sum_{k=1}^{n-1}{G_k(n-k)} = [n/2]$, so $P_n = P_{n-1} + [n/2]$.
\end{solution}
\end{document}

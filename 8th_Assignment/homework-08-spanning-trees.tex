\documentclass[12pt,a4]{article}




\usepackage{graphicx,amsmath,amssymb,amsthm, boxedminipage,xcolor}

\usepackage{algorithm}
\usepackage{algpseudocode}


\newtheorem{theorem}{Theorem}[section]
\newtheorem{proposition}[theorem]{Proposition}
\newtheorem{lemma}[theorem]{Lemma}
\newtheorem{corollary}[theorem]{Corollary}
\newtheorem{definition}[theorem]{Definition}

\newtheorem*{theorem*}{Theorem}
\newtheorem*{lemma*}{Lemma}
\newtheorem*{proposition*}{Proposition}


\newtheorem{exercise}[theorem]{Exercise}
\newtheorem{exerciseD}[theorem]{*Exercise}
\newtheorem{exerciseDD}[theorem]{**Exercise}

\let\oldexercise\exercise
\renewcommand{\exercise}{\oldexercise\normalfont}


\newcommand{\E}{\mathbb{E}}
\newcommand{\scalar}[2]{\ensuremath{\langle #1, #2\rangle}}
\newcommand{\floor}[1]{\left\lfloor #1 \right\rfloor}
\newcommand{\ceil}[1]{\left\lceil #1 \right\rceil}
\newcommand{\norm}[1]{\|#1\|}
\newcommand{\pfrac}[2]{\left(\frac{#1}{#2}\right)}
\newcommand{\nth}[1]{#1^{\textsuperscript{th}}}
\newcommand{\core}{\textnormal{core}}





\newcommand{\poly}{\textnormal{poly}}
\newcommand{\quasipol}{\textnormal{quasipol}}
\newcommand{\ssubexp}{\textnormal{stronglySubExp}}
\newcommand{\wsubexp}{\textnormal{weaklySubExp}}
\newcommand{\simplyexp}{\textnormal{E}}
\newcommand{\expo}{\textnormal{Exp}}



\newcommand{\N}{\mathbb{N}}
\newcommand{\nn}{\mathbb{N}_0^n}
\newcommand{\R}{\mathbb{R}}
\newcommand{\Z}{\mathbb{Z}}


\definecolor{darkgreen}{rgb}{0,0.6,0}


\date{}

\title{
  Mathematical Foundations \\of \\Computer Science\\
  \vspace{3mm}
{\normalsize CS 499,	Shanghai Jiaotong University,  Dominik Scheder}
}

\begin{document}

\maketitle

%\begin{quotation}
%  You are welcome to discuss the exercises in the discussion
%  forum. Please take them serious. Doing the exercises is as important
%  than watching the videos.
%
%  I intentionally included very challenging exercises and marked them
%  with one or two ``$*$''. No star means you should be able to solve
%  the exercises without big problems once you have understood
%  the material from the video lecture. One star means it requires 
%  significant additional thinking. Two stars means it is not 
%  unlikely that you will fail to solve them, even once you have understood
%  the material and thought a lot about the exercise. Don't feel bad
%  if you fail. Failure is part of learning.
%
%  This is the first time this course is online. Thus there might be mistakes
%  (typos or more serious conceptual mistakes) in the exercises. I will be 
%  grateful if you point them out to me!
%\end{quotation}


\setcounter{section}{7}

\section{Spanning Trees}



\begin{itemize}
 \item Homework assignment published on Monday, 2018-04-16
 \item Submit questions and first solution by Sunday, 2018-04-22, 12:00, by email to me and the TAs.
 \item Submit your final solution by Sunday, 2018-04-29.
\end{itemize}

\subsection{Minimum Spanning Trees}


Throughout this assignment, let $G = (V,E)$ be a connected graph
and $w: E \rightarrow \R^+$ be a weight function.\\

\begin{exercise}
    Prove the inverse of the cut lemma: 
      If $X$ is good, $e \not \in X$, and $X \cup e$ is good,
      then there is a cut $S, V\setminus S$ such that (i) no
      edge from $X$ crosses this cut and (ii) $e$ is a minimum
      weight edge of $G$ crossing this cut.
\end{exercise}
\begin{proof}
$X$ is good and $X\cup e$ is good, means that there's a minimum spanning tree $T$ such that $X\subseteq E(T)$ and $X\cup e \subseteq E(T)$. Also, $X$ contains two or more components. 

\begin{itemize}
\item Proof by contradiction that there exists a cut $V=S\cup \overline{S}$ of $X$ such that no edge from $X$ crosses this cut but $e$ crosses this cut:

If $e$ does not cross any cut of $X$, meaning that the two vertices that $e$ connected is within the same connected component of $X$, which leads to a cycle in this component in $X\cup e$. Then it can't be good, a contradiction.
\item Proof by contradiction that $e$ is a minimum weight edge of G crossing this cut:

If $e$ is not the minimum weight edge of G crossing this cut, which means that there exists a edge $e'$ such that $w(e')<w(e)$. Moreover, $w(X\cup e')< w(X\cup e)$. Consider a tree $T'$ such that $E(T')=(E(T)\setminus e)\cup e'$. $T$ is a spanning tree, so that there are exactly 2 connected components if the edge $e$ is subtracted from it. Then by adding the edge $e'$, $T'$ is also a spanning tree since no circles are incidented. However $w(T)-w(T')=w(e)-w(e')>0$, which means that $w(T')<w(T)$. But $T$ is the minimum spanning tree of $G$, thus a contradiction.
\end{itemize}

\end{proof}

\begin{definition}
  For $c \in \R$ and a weighted graph $G = (V,E)$, let
  $G_c := (V, \{e \in E \ | \ w(e) \leq c\})$. That is, $G_c$ is the
  subgraph of $G$ consisting of all edges of weight at most $c$.
\end{definition}

\begin{lemma}
  Let $T$ be a minimum spanning tree of $G$, and let $c \in \R$.  Then
  $T_c$ and $G_c$ have exactly the same connected components.  (That
  is, two vertices $u,v \in V$ are connected in $T_c$ if and only if
  they are connected in $G_c$).
  \label{lemma-CC}
\end{lemma}

\begin{exercise}
     Illustrate Lemma~\ref{lemma-CC} with an example!
\end{exercise}
\begin{solution}
\end{solution}
For a figure as below: 
\begin{center}
\includegraphics[width=0.4\textwidth]{figures/txy.pdf}
\end{center}
The $MST$ $T$ would be
\begin{center}
\includegraphics[width=0.4\textwidth]{figures/t.pdf}
\end{center}
Assume that $c=2$, then $G_2$ is
\begin{center}
\includegraphics[width=0.4\textwidth]{figures/gc.pdf}
\end{center}
And $T_2$ is 
\begin{center}
\includegraphics[width=0.4\textwidth]{figures/tc.pdf}
\end{center}
Lemma 8.3, obviously, holds for the example above.
\begin{exercise}
     Prove the lemma.
\end{exercise}
\begin{proof}
The proof is given in $2$ directions as below:
\end{proof}
\begin{itemize}
\item two vertices $u,v \in V$ are connected in $T_c$ if they are connected in $G_c$:
\begin{itemize}
\item $\forall e \in E,$ $c\ge w(e)$. Then $T_c$ is the spanning tree which connects every 2 vertices. And $G_c$ is the $G$ itself. The lemma obviously holds.
\item $\exists e \in E,$  $c< w(e)$. Proof by contradiction, if there exist 2 vertices $u,v$ that are connected in $G_c$, but are not connected in $T_c$.
Obviously, $u,v$ are connected in $T$. Suppose the connecting path is $(e_1,e_2,...,e_k)$. There exists an edge $e_i\in E(T)$, $1\le i\le k$, such that $w(e_i)>c$. For the only cut $V$ in the 2 connected components of $T'$, where $E(T')=E(T)\setminus e_i$. According to the cut lemma, $e_i$ is the minimum weight edge that connecting the 2 components. Hence, there can not exist a path in which all the edges' weights are less than $c$ in $G$. So $u,v$ can not be connected in $G_c$, a contradiction.
\end{itemize}
\item two vertices $u,v \in V$ are connected in $G_c$ if they are connected in $T_c$:

The proof of this direction is much simpler. If $u,v$ are connected in $T_c$, then there must exists a path $(e_1,e_2,...,e_n)$ connecting $u$ and $v$, in which every edge weight is no more than $c$. Further, all the edges in this path would be in $G_c$, which leads to a path connecting $u,v$ in $G_c$ as well.
\end{itemize}


\begin{definition}
  For a weighted graph $G$, let $m_c(G) := | \{ e \in E(G) \ | \ w(e) \leq c\}|$, i.e.,
  the number of edges of weight at most $c$ (so $G_c$ has $m_c(G)$ edges).
\end{definition}

\begin{lemma}
  Let $T, T'$ be two minimum spanning trees of $G$. Then
  $m_c(T) = m_c(T')$.
  \label{lemma-weight-multiplicity}
\end{lemma}

\begin{exercise}
Illustrate Lemma~\ref{lemma-weight-multiplicity} with an example!
\end{exercise}
\begin{solution}
Using the same example in \textbf{exercise 8.4}, the minimum spanning tree is
\begin{center}
\includegraphics[width=0.4\textwidth]{figures/mst1.pdf}
\end{center}
\begin{center}
\includegraphics[width=0.4\textwidth]{figures/mst2.pdf}
\end{center}
The lemma, obviously, holds.
\end{solution}
\begin{exercise}
Prove the lemma.
\end{exercise}
\begin{proof}
$T$ and $T'$ both has exactly $|V|-1$ edges.

Assume that $E(T)=(e_1,...,e_k),$ in an order such that $ w(e_1)\le w(e_2)\le...\le w(e_k)$, and $E(T')=(e'_1,...,e'_k), $ in an order such that $w(e'_1)\le w(e'_2)\le...\le w(e'_k)$ and $ k=|V|-1$.

First, we proof that the weight sequence $w(e_1)w(e_2)...w(e_k)$ and 
\\$w(e'_1)w(e'_2)...w(e'_k)$ are exactly the same.

Consider constructing the minimum spanning tree by Kruskal’s Algorithm. In this approach, each time when you want to expand your subgraph by adding an edge, you only care about the weight of an edge but not the edge, say $(v_i, v_k)$, itself. So, each time, you will always get the exactly weight sequence, however, not always the same MST.

Therefore, for two MST trees of the same weight sequence, the lemma obviously holds.
\end{proof}

\begin{exercise}
  Suppose no two edges of $G$ have the same weight. 
  Show that $G$ has exactly one minimum spanning tree!
\end{exercise}
\begin{proof}
Assume there are two minimum spanning trees $T_1$ and $T_2$ of a graph. Consider the edge of the minimum weight among all edges that are contained in \textit{exactly one} of the two minimum spanning trees. Without loss of generality, this edge $e_1$ appears only in $T_1$. $T_2\cup\{e_1\}$ has a cycle, in which there exists an edge $e_2$ only contained in $T_2$, not in $T_1$. And we have $w(e_1) < w(e_2)$, thus $T_2\cup\{e_1\}\slash\{e_2\}$ is a spanning tree of smaller total weight than $T_2$. $T_2$ is not a minimum spanning tree. Therefore, the graph has a unique minimum spanning tree.
\end{proof}



\subsection{Counting Special Functions}

In the video lecture, we have seen a connection between functions $f: V \rightarrow V$
and trees on $V$. We used this to learn something about the number of such trees.
Here, we will go in the reverse direction: the connection will actually teach us a bit
about the number of functions with a special structure.\\

Let $V$ be a set of size $n$. We have learned that there are $n^n$ functions $f: V \rightarrow V$.
For such a function we can draw an ``arrow diagram" by simply drawing an arrow from
$x$ to $f(x)$ for every $V$. For example, let $V = \{0,\dots,7\}$ and $f(x) := x^2 \mod 8$.
The arrow diagram of $f$ looks as follows:
\begin{center}
\includegraphics[width=0.3\textwidth]{figures/arrow-diagram.pdf}
\end{center}
The {\em core} of a function is the set of elements lying on cycles in such a diagram.
For example, the core of the above function is $\{0,1\}$. Formally, the core of $f$ is the 
set
\begin{align*}
%\core(f) = 
\{x \in V \ | \ \exists k \geq 1 f^{(k)}(x) = x \}
\end{align*}
where $f^{(k)}(x) = f( f( \dots f(x) \dots ))$, i.e., the function $f$ applied $k$ times 
iteratively to $x$.

\begin{exercise}
Of the $n^n$ functions from $V$ to $V$, how many have a core of size $1$? 
Give an explicit formula in terms of $n$.
\end{exercise}
\begin{solution}
One mapping from $V$ to $V$ has a corresponding mapping graph and a corresponding vertebrate in a spanning tree of $(V,E)$.

Hence, how many functions have a core of size $1$ is equal to how many vertebrates generate a mapping graph of exactly one cycle in all the spanning tree with vertices $V$.

The number of spanning tree is $n^{n-2}$.

For each spanning tree,There are $n$ choices to form a function of only $1$ core, that is , by choosing both the \textit{HEAD} and \textit{BUTT} exactly the same. 

In total, there are $n^{n-2}\times n= n^{n-1}$ functions that have a core of size 1.

\end{solution}

\begin{exercise}
How many have a core of size $2$ that consists of two $1$-cycles? By this we mean
that $\core(f) = \{x,y\}$ with $f(x) = x$ and $f(y) = y$.
\end{exercise}

\begin{solution}
The number of spanning tree is $n^{n-2}$.

For each spanning tree, we choose 2 different elements, connected by the same edge, as \textit{HEAD} and \textit{BUTT} and let them form a cycle to itself in the corresponding mapping graph. Hence, there are $(n-1)$ choices.

In total, there are $n^{n-2}\times(n-1)$ functions.
\end{solution}
\textbf{Hint.} For the previous two exercises, you need to use the link between functions
 $f: [n]\rightarrow [n]$ and 
vertebrates $(T,h,b)$ from the video lecture. 

%一个spanning tree 对应n^2个vertebrates 一个vertebrates 对应一个permutation。




\subsection{Counting Trees with Pr\"ufer Codes}

In the video lecture, we have seen Cayley's formula, stating that there are exactly
$n^{n-2}$ trees on the vertex set $[n]$. We showed a proof using 
{\em vertebrates}. For this homework, read Section 7.4 of the textbook, titled
``A proof using the Pr\"ufer code''.


\begin{exercise}
  Let $V = \{1,\dots,9\}$ and consider the code $(1,3,3,2,6,6,1)$. Reconstruct a tree
  from this code. That is, find a tree on $V$ whose Pr\"ufer code is $(1,3,3,2,6,6,1)$.
\end{exercise}

\begin{solution}
\begin{figure}[h]
    \centering
    \includegraphics[width=6cm]{figures/8_13.png}
\end{figure}
\end{solution}

\begin{exercise}
  Let $\mathbf{p} = (p_1,p_2,\dots,p_{n-2})$ be the Pr\"ufer  code of some tree $T$ on $[n]$.
  Find a way to quickly determine the degree of vertex $i$ only by looking
  at $\mathbf{p}$ and not actually constructing the tree $T$.  
  In particular, by looking at $\mathbf{p}$, what are the leaves of $T$?
\end{exercise}

\begin{solution}
  The degree of vertex $i$ equal to the times $i$ appearing in $\mathbf{p}$ plus $1$. The leaves of $T$ are those vertex which do not appear in the Pr\"ufer code.
\end{solution}


\begin{exercise}
  Describe which tree on $V = [n]$  has the
  \begin{enumerate}
  \item Pr\"ufer code $(1,1,\dots,1)$.
  \item Pr\"ufer code $(1,2,3,\dots, n-2)$.
  \item Pr\"ufer code $(3,4,5,\dots, n)$.
  \item Pr\"ufer code $(n, n-1, n-2,\dots,4,3)$.
  \item Pr\"ufer code $(n-2,n-3,\dots,2,1)$.
  \item Pr\"ufer code $(1,2,1,2,\dots,1,2)$ (assuming $n$ is even).
 \end{enumerate}
 Justify and explain your answers.
\end{exercise}

\begin{solution}
  \begin{enumerate}
    \item In the tree, the vertex $1$ is directly connected to all other vertices. In the encode process, when tearing off a vertex, $1$ will always be added to the Pr\"ufer code.
    \item The tree is a path, start from $n-1$, to $1,2,\ldots,n-3,n-2,n$. In the encode process, we firstly tear off $n-2$, adding $1$ to the code. After that, we tear off $1,2,\ldots,n-3$ in order, adding $2,3,\ldots,n-2$ to the code. When the process is finished, the remaining vertices are $n-2$ and $n$.
    \item The tree is a path, whose vertices from one endpoint to the other endpoint are $1,3,\ldots, n-1, n, n-2, \ldots,4 ,2$ ($n$ is even) or $1, 3,\ldots, n-2,n,n-1,\ldots,4,2$ ($n$ is odd). In the encode process, we always tear off one endpoint of the remaining tree, then tear off the other endpoint, and repeat the procedure until there exist only two vertices. It'obvious that we will get the Pr\"ufer code $(3,4,5,\dots, n)$.
    \item The tree is a path. The verices start from one endpoint to the other are $1,n,3,\ldots,n-2,n-1,2$. We tear off $1,2,n-1,n-2,\ldots,5,4$ and encode the Pr\"ufer code with $n,n-1,n-2,\ldots,4,3$.
    \item The tree is a path. From one endpoint to the other endpoint, the vertices are $n-1,n-2,\ldots,2,1,n$. $n$ will never be torn off, so we tear off $n-1,n-2,\ldots,3,2$, adding $n-2,n-3,\ldots,2,1$ to the Pr\"ufer code.
    \item $1$ and $2$ are connected by an edge. $3,5,\ldots,n-1$ are directly connected to $1$, while $4,6,\ldots,n$ are connected to $2$. We tear off $3,4,5,6,\ldots,n$ in order, so the code is $(1,2,1,2,\dots,1,2)$.
  \end{enumerate}
\end{solution}
  

The next two exercises use a bit of probability theory. Suppose we want to sample a random
tree on $[n]$. That is, we want to write a little procedure (say in Java) that uses randomness and outputs
a tree $T$ on $[n]$, where each of the $n^{n-2}$ trees has the same probability of 
appearing. 

\begin{exercise}
   Sketch how one could write such a procedure. Don't actually write program code, just
   describe it informally. You can assume you have access to a random generator
   \texttt{randomInt($n$)} that returns a function in $\{1,\dots,n\}$ as well
   as \texttt{randomReal()} that returns a random real number from the interval $[0,1]$.
\end{exercise}

\begin{solution}
  We use \texttt{randomInt(n)} to generate a random Pr\"ufer code with $n-2$ digits. For each digit of the code, we generate a random integer from $1$ to $n$. By decoding the Pr\"ufer code, it returns a random tree $T$.
\end{solution}

Clearly, a tree $T$ on $[n]$ has at least $2$ and at most $n-1$ leaves. But how 
many leaves does it have on average? For this, we could use your tree sampler from the previous
exercise, run it 1000 times and compute the average. However, it would be much nicer to have 
a closed formula.

\begin{exercise}
  Fix some vertex $u \in [n]$. If we choose a tree $T$ on $[n]$ uniformly at random,
  what is the probability that $u$ is a leaf?
   What is the expected number of leaves of $T$?
\end{exercise}

\begin{solution}
  $u$ is a leaf, in other words, $u$ does not appear in the Pr\"ufer code. The probability is $(n-1/n)^n$. The expected number of leaves of $T$ is $n(n-1/n)^n$.
\end{solution}

\begin{exercise}
  For a fixed vertex $u$, what is the probability that $u$ has degree $2$?
\end{exercise}

\begin{solution}
  For a fixed vertex $u$, if $u$ has degree $2$, $u$ appears in the Pr\"ufer code exactly once. The probability is $n(n-1/n)^{n-1}(1/n) = (n-1/n)^{n-1}$.
\end{solution}







\end{document}

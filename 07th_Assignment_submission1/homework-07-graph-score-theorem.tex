\documentclass[12pt,a4]{article}




\usepackage{graphicx,amsmath,amssymb,amsthm, boxedminipage,xcolor}

\usepackage{algorithm}
\usepackage{algpseudocode}


\newtheorem{theorem}{Theorem}[section]
\newtheorem{proposition}[theorem]{Proposition}
\newtheorem{lemma}[theorem]{Lemma}
\newtheorem{corollary}[theorem]{Corollary}
\newtheorem{definition}[theorem]{Definition}

\newtheorem*{theorem*}{Theorem}
\newtheorem*{lemma*}{Lemma}
\newtheorem*{proposition*}{Proposition}


\newtheorem{exercise}[theorem]{Exercise}
\newtheorem{exerciseD}[theorem]{*Exercise}
\newtheorem{exerciseDD}[theorem]{**Exercise}

\let\oldexercise\exercise
\renewcommand{\exercise}{\oldexercise\normalfont}


\newcommand{\E}{\mathbb{E}}
\newcommand{\scalar}[2]{\ensuremath{\langle #1, #2\rangle}}
\newcommand{\floor}[1]{\left\lfloor #1 \right\rfloor}
\newcommand{\ceil}[1]{\left\lceil #1 \right\rceil}
\newcommand{\norm}[1]{\|#1\|}
\newcommand{\pfrac}[2]{\left(\frac{#1}{#2}\right)}
\newcommand{\nth}[1]{#1^{\textsuperscript{th}}}
\newcommand{\core}{\textnormal{core}}





\newcommand{\poly}{\textnormal{poly}}
\newcommand{\quasipol}{\textnormal{quasipol}}
\newcommand{\ssubexp}{\textnormal{stronglySubExp}}
\newcommand{\wsubexp}{\textnormal{weaklySubExp}}
\newcommand{\simplyexp}{\textnormal{E}}
\newcommand{\expo}{\textnormal{Exp}}



\newcommand{\N}{\mathbb{N}}
\newcommand{\nn}{\mathbb{N}_0^n}
\newcommand{\R}{\mathbb{R}}
\newcommand{\Z}{\mathbb{Z}}


\definecolor{darkgreen}{rgb}{0,0.6,0}


\date{}

\title{
  Mathematical Foundations \\of \\Computer Science\\
  \vspace{3mm}
{\normalsize CS 499,	Shanghai Jiaotong University,  Dominik Scheder}
}

\begin{document}

\maketitle

%\begin{quotation}
%  You are welcome to discuss the exercises in the discussion
%  forum. Please take them serious. Doing the exercises is as important
%  than watching the videos.
%
%  I intentionally included very challenging exercises and marked them
%  with one or two ``$*$''. No star means you should be able to solve
%  the exercises without big problems once you have understood
%  the material from the video lecture. One star means it requires 
%  significant additional thinking. Two stars means it is not 
%  unlikely that you will fail to solve them, even once you have understood
%  the material and thought a lot about the exercise. Don't feel bad
%  if you fail. Failure is part of learning.
%
%  This is the first time this course is online. Thus there might be mistakes
%  (typos or more serious conceptual mistakes) in the exercises. I will be 
%  grateful if you point them out to me!
%\end{quotation}





\setcounter{section}{6}

\section{The Graph Score Theorem}



\begin{itemize}
 \item Homework assignment published on Monday, 2018-04-09.
 \item Submit questions and first solution by Sunday, 2018-04-15, 12:00 by
 email to dominik.scheder@gmail.com and the TAs.
 \item You will receive feedback by Wednesday, 2018-04-18.
 \item Submit your final solution by Sunday, 2018-04-22 to me and the two TAs.
\end{itemize}



\begin{exercise}
  Describe, in simple sentences with a minimum of mathematical formalism, (1) the score
  of a graph, (2) what the graph score theorem is, (3) the idea of the
  graph score algorithm, (4) where the difficult part of its proof is.
  Imagine you have a friend who does not take this class, and think about how to answer
  the above questions to them.
\end{exercise}

\begin{solution}
  \quad \newline
\begin{enumerate}
\item The score of a graph is the increasing sequence of degrees of all vertexes.
\item Give a increasing sequence of degrees $ D = (d_1, d_2, d_3 \dots d_n)$ with $d_i$ denoting its element. Consider another sequnces $D^{'}$ of size $n-1$ whose element denoted as $d_i^{'}$ with:\\
\begin{displaymath}
d_i^{'} = \left\{ \begin{array}{ll}
d_i & i < n - d_n\\
d_i - 1 & i \ge n - d_n
\end{array}\right.
\end{displaymath}
\end{enumerate}
$D$ is a graph score if and only if $D^{'}$ is a graph score.

\item The idea of graph score algorithm is recursively using theorem to simplify the degree sequnce until it's easy to judge whether the sequnce is a graph score.

\item The difficult part of the proof is the reverse implication i.e. if $D$ is a graph score then $D^{'}$ is a graph score. Since we can't just reverse the constrution process how we get the graph of $D$ from the graph of $D^{'}$.
\end{solution}


\subsection{Alternative Graphs}

Now we will look at different notions of graphs. As defined in class and in the video
lectures, a graph is a pair $G = (V,E)$ where $V$ is a (usually finite) set, called the {\em vertices},
and $E \subseteq {V \choose 2}$, called the set of {\em edges}.

\paragraph{Multigraphs.}
A {\em multigraph} is like a graph, but you can have several parallel edges between
two vertices. You cannot, however, have self-loops. That is, there cannot
be an edge from $u$ to $u$ itself. This is an example of a multigraph:

\begin{center}
  \includegraphics[width=0.25\textwidth]{figures/multigraph-score.pdf}
\end{center}

We can define degree and score for multigraphs, too. For example, this multigraph
has score $(4,4,2)$. Obviously no graph
can have this score.

\begin{exercise}
  State a score theorem for multigraphs. That is, something like
  \begin{theorem}[Multigraph Score Theorem]
     Let $(a_1,\dots,a_n) \in \N_0^n$. There is a multigraph
     with this score if and only if \texttt{\textup{<fill in some simple criterion here>}}.
  \end{theorem}

  \textbf{Remark.} This is actually
  simpler than for graphs.
\end{exercise}

\begin{solution}
  Let $(a_1,\dots,a_n) \in \N_0^n$. There is a multigraph with this score if and only if 1) $\sum_{i = 1}^n a_i$ is even and 2) $\sum_{i=1}^{n-1} a_i \ge a_n$.
\end{solution}

\begin{exercise}
  Prove your theorem.
\end{exercise}

\begin{proof}
First, if $\mathbf{a} = (a_1,\dots,a_n) \in \N_0^n$ is multigraph score then it obviously satisfy the criterion according to Handshake Lemma.\\
To prove the reverse implication, we use induction theorem.\\
\textbf{Hypothesis:} Any sequcence $\mathbf{a} = (a_1^{'},\dots,a_{n-1}^{'})$ is a multigraph score satisfying both criterions, then it has corresponding multigraph with score $\mathbf{a}$.\\
\textbf{Base case:} $\mathbf{a^{'}} = (a_1^{'}, a_2^{'})$ meeting the conditons obviously has corresponding multigraph score $\mathbf{a} = (a_1, a_2, a_3)$.\\
\textbf{Induction proof:} For any nodecreasing order sequnce $\mathbf{a} = (a_1, a_2, \dots a_n)$ satisfying both criterions, remove the node with degree $a_1$ in $\mathbf{a}$ to get $\mathbf{a^{'}} = (a_2, a_3, \dots a_{n}^{'})$ where $a_n^{'} = a_n - a_1$. \\
If $a_n^{'}$ is still the largest one, then $\mathbf{a^{'}}$ is obviously a multigraph score. Therefore $\mathbf{a}$ is also a multigraph score according to induction theorem. \\
If $a_n^{'}$ is not the largest one, which means $a_{n-1}$ is the largest one. Since $a_n > a_{n-1}$ and $a_n^{'} + a_2$ is at least $a_n + a_2 - a_1 > a_n \ge a_{n-1}$, which implies $\mathbf{a^{'}}$ is multigraph score.\\
Then, we can conclude that the theorem is correct.
\end{proof}
% \textbf{Induction Hypothesis}: a sequence \textbf{a} = $(a_1,...,a_{n-1})(a_1\leq a_2\leq ... \leq a_{n-1})$ satisfying condition (1) and (2) always has a corresponding multigraph.\\
% \textbf{Base Case}: n=2, \textbf{a} = $(a_1,a_2)\in$ \emph{$N_0^n$} $(a_1 = a_2)$. For two vertices, we can easily find out the corresponding multi-graph.\\
% \textbf{Induction Step}: Given a score with $n$ vertices, do as follows to reduce the size of the problem to $n-1$.\\
% Denote by $V_i$ the corresponding vertex with degree $a_i$. Given a score \textbf{a} = $(a_1,...,a_n) \in $ \emph{$N_0^n$} $(a_1\leq a_2\leq ... \leq a_n)$, do:
% \begin{enumerate}
%   \item Connect $a_1$ edges between $V_1$ and $V_n$, and thus "eliminate" $a_1$. we now have a new score \textbf{$a'$} = $(a_2,...,a_{n-1},a_n-a_1)$
%   \item Sort the newly formed score and we have \textbf{$a'$} = $(a_1^{'}...,a_{n-1}^{'})$.
%   \item Claim: Condition (1) and (2) still hold for the newly formed score \textbf{$a'$}. With this claim, we know that \textbf{$a'$} is feasible for a multigraph based on the induction. Then, what we have to do is simply a matter of adding $a_1$ edges between $V_1$ and $V_n$. and this is your multigraph.
% \end{enumerate}
% This concludes the proof of the induction, and the whole theorem.\\\\
% \textbf{Proof to the claim}:\\
% For \textbf{$a'$} = $(a_2,...,a_{n-1},a_n-a_1)$,
% \begin{itemize}
%   \item if $a_n-a_1\geq a_{n-1}$, we have $a_2+a_3+...+a_{n-1}\geq a_n-a_1$, which satisfies condition (2). Beside, since the sum of all elements is even, changing $a_1$ to its negation $-a_1$ will not affect its parity. Namely, $-a_1+\sum\limits_{i=2}^n a_i$ is still even. This satisfies condition (1). Hence, our claim follow immediately.
%   \item if $a_n-a_1\leq a_{n-1}$(In other words, $a_{n-1}$ will be the largest element), we know $-a_1+a_2+...+a_{n-2}\geq 0$, and $a_n\geq a_{n-1}$. And it implies that $-a_1+a_2+...+a_{n-2}+a_n \geq a_{n-1}$. Then it's easy to notice that condition (2) still holds for this new score. And condition (1) holds for the same reason above.
% \end{itemize}
% Therefore, we conclude that such a operation will indeed lead to a new score which satisfies the conditions.
% \end{proof}

\newcommand{\wdeg}{\textnormal{wdeg}}

\paragraph{Weighted graphs.} A weighted graph is a graph in which every edge $e$ has a non-negative weight $w_e$.
In such a graph the {\em weighted degree} of a vertex $u$ is $\wdeg(u) = \sum_{ \{u,v\} \in E} w_{\{u,v\}}$.

\begin{center}
  \includegraphics[width=0.25\textwidth]{figures/weighted-graph-score.pdf}
\end{center}

This is an example of a weighted graph, which has score $(3,2,2)$. Obviously no graph
and no multigraph can have this score.

\begin{exercise}
  State a score theorem for weighted graphs. That is, state something like
  \begin{theorem}[Weighted Graph Score Theorem]
     Let $(a_1,\dots,a_n) \in \R_0^n$. There is a weighted graph
     with this score if and only if \texttt{\textup{<fill in some simple criterion here>}}.
  \end{theorem}
  \textbf{Remark.} This
  is actually even simpler.
\end{exercise}
\begin{solution}
  Let $(a_1,\dots,a_n) \in \R_0^n$. There is a weighted graph with this score if and only if $\sum_{i = 1}^{n-1} a_i\ge a_n$.
\end{solution}

\begin{exercise}
  Prove your theorem.
\end{exercise}

\begin{proof}
First, for any weighted graph, we have $\sum_{i = 1}^{n-1}a_i \ge a_n$ since its each edge connected to $a_n$ also connecting a node in $(a_1, \dots a_{n-1})$.\\
Besides, for any sequnce $\mathbf{a} = (a_1,\dots,a_n) \in \R_0^n$ with $\sum_{i = 1}^{n-1}a_i \ge a_n$, we use induction theorem to prove the theorem.\\
\textbf{Hypothesis}: Any sequnce $\mathbf{a_{'}} = (a_1^{'},\dots,a_{n-1}^{'}) \in \R_0^n$ which is weighted graph score has a corresponding sequnce $\mathbf{a} = (a_1,\dots,a_n) \in \R_0^n$ meeting the criterion is a weighted graph score.\\
\textbf{Base case}: $\mathbf{a} = {a_1, a_2}$ with $a_1 \ge a_2$ is a weighted graph score.\\
\textbf{Induction step}: For any nodecreasing order sequnce $\mathbf{a} = (a_1,\dots,a_n) \in \R_0^n$ with $\sum_{i = 1}^{n-1} a_i\ge a_n$, remove vertex with degree $a_1$ and connected edges to get $\mathbf{a_{'}} = (a_2,\dots,a_{n}^{'})$ where $a_n^{'} = a_n - a_1$.\\
If $a_{n}^{'}$ is still the largest element in $\mathbf{a^{'}}$, it's obviously that $\sum_{i = 2}^{n-1}a_i \ge a_n - a_1$, which means $\mathbf{a^{'}}$ is a graph weighted score.\\
If $a_{n}^{'}$ is not the largest element, which means $a_{n-1}$ is the largest one. We have $\sum_{i = 2}^{n-2}a_i + a_n - a_1 \ge a_{n-1}$, so $\mathbf{a^{'}}$ is a graph weighted score.\\
Accordingly, we have the theorem proved.
\end{proof}

\paragraph{Allowing negative edge weights.} Suppose now we allow negative edge weights, like here:
\begin{center}
  \includegraphics[width=0.25\textwidth]{figures/arbitrary-weighted-graph-score.pdf}
\end{center}
This ``graph with real edge weights'' has score $(2,0,0)$. This score is impossible for
graphs, multigraphs, and weighted graphs with non-negative edge weights.


\begin{exercise}
  State a score theorem for weighted graphs when we allow
  negative edge weights. That is, state a theorem like
  \begin{theorem}[Score Theorem for Graphs with Real Edge Weights]
     Let $(a_1,\dots,a_n) \in \R^n$. There is a  graph with real edge weights
     with this score if and only if \texttt{\textup{<fill in some simple criterion here>}}.
  \end{theorem}

 (Score Theorem For Graphs With Real Edge Weights).Let$(a_1,\cdots,a_n)\in \mathbb{R}^n$.There is a graph with real edge weights with this score if and only if
      $$\begin{cases}
           a_n=0&n=1,\\
           a_{n-1}=a_n&n=2,\\
           always\ holds &n\geq 3.
           \end{cases}$$
\end{exercise}

\begin{exercise}
Prove your theorem.
\begin{proof}

\textbf{Induction Hypothesis:} For any given score $\textbf{a}=(a_1,\cdots,a_{n-1},a_{n}) (n\geq3)$, there's always a corresponding weightedgraph whose score is $\textbf{a}$.\\
\textbf{Base Case:}\\
            If n=1, the score must be 0 because there is no edge in the graph.\\
            If n=2, only one edge is connecting these two vertices, and thus, their degrees should be the same..\\
            If $n=3$, denote by W(u,v) the weight of the edge connecting vertex u and v. Then we have
            $$\begin{cases}W(a_1,a_2)=(a_1+a_2-a_3)/2,\\
                       W(a_1,a_3)=(a_1-a_2+a_3)/2,\\
                       W(a_2,a_3)=(a_2+a_3-a_1)/2.
             \end{cases}$$
            Since $a_1,a_2,a_3 \in \textbf{R}$, there is always a solution to this equation array, which corresponds to a graph with real weights. \\
\textbf{Induction Step:}\\
        If $n>3$,we can reduce the size of the problem to a new score $\textbf{a'}=(a'_1,\cdots,a'_{n-1}$) by the following operations:
        $$a'_i=\begin{cases}
        a_i&1\leq i<n-1,\\
        a_i-a_{i+1}&i=n-1.
        \end{cases}$$\\
        Then, we could apply the induction to this newly formed score, i.e., we could construct a weightedgraph with $\textbf{a'}$.\\ And we just connect $n^{th}$ vertex to the $(n-1)^{th}$ with an edge of weight $a_n$. And that's your weightedgraph.\\
        That concludes this proof.
\end{proof}
\end{exercise}
\begin{exercise}
   For each student ID $(a_1,\dots,a_n)$ in your group, check whether
   this is (1) a graph score, (2) a multigraph score, (3) a weighted graph score, or
   (4) the score of a graph with real edge weights.

   Whenever the answer is {\em yes}, show the graph, when it is {\em no},
   give a short argument why.
\end{exercise}

\paragraph{Example Solution.} My work ID is 50411.
This is a weighted graph score, as shown by this picture:
\begin{center}
\includegraphics[width=0.4\textwidth]{figures/graph-50411.pdf}
\end{center}
This settles (3). It is not a multigraph score, because BLABLABLA. I won't give more details, as it might
give too many hints about Exercise 7.2. Alright, this settles (2). Note that I do {\em not} need to answer
(4), as this is already answered by (3). Neither do I need to answer (1), as a ``no'' for (2) implies
a ``no'' for (1).

\begin{enumerate}
	\item 515021910302
	\begin{itemize}
	\item a graph score:There are only eight vertices but there is one 9, so it’s impossible for graph score.
	\item a multigraph score: no, the sum of the degrees can not be an odd number.
	\item 
\begin{center}
\includegraphics[width=0.4\textwidth]{figures/ccc.pdf}
\end{center}
	The figure above settles for (3) and (4).
	\end{itemize}
	\item 516021910003
	\begin{itemize}
	\item a graph score:There are only seven vertices but there is one 9, so it’s impossible for graph score.
\begin{center}
\includegraphics[width=0.4\textwidth]{figures/2.pdf}
\end{center}
	The figure above settles for (2), (3) and (4).
	\textit{P.S.for a multigraph, simply ignoring the weight value, and for a weighted graph, simply ignoring the multi-edges between 2 vertices and see them as only 1 edge.}
		\end{itemize}
	\item 516021910725
		\begin{itemize}
	\item a graph score:There are only seven vertices but there is one 9, so it’s impossible for graph score.
	\item a multigraph score: no, the sum of the degrees can not be an odd number.
		\item 
\begin{center}
\includegraphics[width=0.4\textwidth]{figures/3.pdf}
\end{center}
	The figure above settles for  (3) and (4).
		\end{itemize}
	\item 515021910053
	\begin{itemize}
	\item a graph score:There are only eight vertices but there is one 9, so it’s impossible for graph score.
\begin{center}
\includegraphics[width=0.4\textwidth]{figures/txy-1.pdf}
\end{center}
		\end{itemize}
\begin{center}
\includegraphics[width=0.4\textwidth]{figures/txy.pdf}
\end{center}
	The figures above settles for (2), (3) and (4).

	\item 516021910127
		\begin{itemize}
	\item a graph score:According to the graph score theorem: $(001111225679)$\\-->$(00000011456)$, since in the second step it is obvious that the score cannot denote a graph. Hence, it’s impossible for graph score.
	\item a multigraph score: no, the sum of the degrees can not be an odd number.
\begin{center}
\includegraphics[width=0.4\textwidth]{figures/4.pdf}
\end{center}
	The figure above settles for (3) and (4).

		\end{itemize}
\end{enumerate}




\end{document}

\documentclass[12pt,a4]{article}







\usepackage{graphicx,amsmath,amssymb,amsthm, boxedminipage,xcolor}

\usepackage{algorithm}
\usepackage{algpseudocode}


\newtheorem{theorem}{Theorem}[section]
\newtheorem{proposition}[theorem]{Proposition}
\newtheorem{lemma}[theorem]{Lemma}
\newtheorem{corollary}[theorem]{Corollary}
\newtheorem{definition}[theorem]{Definition}

\newtheorem*{theorem*}{Theorem}
\newtheorem*{lemma*}{Lemma}
\newtheorem*{proposition*}{Proposition}


\newtheorem{exercise}[theorem]{Exercise}
\newtheorem{exerciseD}[theorem]{*Exercise}
\newtheorem{exerciseDD}[theorem]{**Exercise}

\let\oldexercise\exercise
\renewcommand{\exercise}{\oldexercise\normalfont}


\newcommand{\E}{\mathbb{E}}
\newcommand{\scalar}[2]{\ensuremath{\langle #1, #2\rangle}}
\newcommand{\floor}[1]{\left\lfloor #1 \right\rfloor}
\newcommand{\ceil}[1]{\left\lceil #1 \right\rceil}
\newcommand{\norm}[1]{\|#1\|}
\newcommand{\pfrac}[2]{\left(\frac{#1}{#2}\right)}
\newcommand{\nth}[1]{#1^{\textsuperscript{th}}}
\newcommand{\core}{\textnormal{core}}





\newcommand{\poly}{\textnormal{poly}}
\newcommand{\quasipol}{\textnormal{quasipol}}
\newcommand{\ssubexp}{\textnormal{stronglySubExp}}
\newcommand{\wsubexp}{\textnormal{weaklySubExp}}
\newcommand{\simplyexp}{\textnormal{E}}
\newcommand{\expo}{\textnormal{Exp}}



\newcommand{\N}{\mathbb{N}}
\newcommand{\nn}{\mathbb{N}_0^n}
\newcommand{\R}{\mathbb{R}}
\newcommand{\Z}{\mathbb{Z}}


\definecolor{darkgreen}{rgb}{0,0.6,0}


\date{}

\title{
  Mathematical Foundations \\of \\Computer Science\\
  \vspace{3mm}
{\normalsize CS 499,	Shanghai Jiaotong University,  Dominik Scheder}
}

\begin{document}

\maketitle

%\begin{quotation}
%  You are welcome to discuss the exercises in the discussion
%  forum. Please take them serious. Doing the exercises is as important
%  than watching the videos.
%
%  I intentionally included very challenging exercises and marked them
%  with one or two ``$*$''. No star means you should be able to solve
%  the exercises without big problems once you have understood
%  the material from the video lecture. One star means it requires 
%  significant additional thinking. Two stars means it is not 
%  unlikely that you will fail to solve them, even once you have understood
%  the material and thought a lot about the exercise. Don't feel bad
%  if you fail. Failure is part of learning.
%
%  This is the first time this course is online. Thus there might be mistakes
%  (typos or more serious conceptual mistakes) in the exercises. I will be 
%  grateful if you point them out to me!
%\end{quotation}




\setcounter{section}{1}


\begin{itemize}
 \item Homework assignment published on Monday, 2018-03-05.
 \item Work on it and submit a first solution or questions by Sunday, 2018-03-11, 12:00 by
 email to me and the TAs.
 \item You will receive feedback by Wednesday, 2018-03-14.
 \item Submit your final solution by Sunday, 2018-03-18 to me and the TAs.
\end{itemize}



\section{Partial Orderings}



\subsection{Equivalence Relations as a Partial Ordering}

An equivalence relation $R \subseteq V \times V$ is basically the same as a partition
of $V$. A {\em partition} of $V$ is a set $\{V_1,\dots,V_k\}$ where
(1) $V_1 \cup \dots \cup V_k = V$ and (2) the $V_i$ are pairwise disjoint,
i.e., $V_i \cap V_j  = \emptyset$ for $1 \leq i < j \leq k$. For example,
$\{ \{1\}, \{2,3\}, \{4\} \}$ is a partition of $\{1,2,3,4\}$ but
$\{ \{1\}, \{2,3\}, \{1,4\}\}$ is not.



\begin{exercise}
  Let $E_4$ be the set of all equivalence relations on $\{1,2,3,4\}$. Note that
  $E_4$ is ordered by set inclusion, i.e.,
  \begin{align*}
     (E_4, \{ (R_1,R_2) \in E_4 \times E_4 \ | \ R_1 \subseteq R_2 \} )
  \end{align*}
  is a partial ordering.
  \begin{enumerate}
    \item Draw the Hasse diagram of this partial ordering in a  nice way.
    \item What is the size of the largest chain?
    \item What is the size of the largest antichain?
  \end{enumerate}
  \end{exercise}


\subsection{Chains and Antichains}

Define the partially ordered set $(\nn, \leq)$ as follows:
$x \leq y$ if $x_i \leq y_i$ for all $1 \leq i \leq n$. For example,
$(2,5,4) \leq (2,6,6)$ but $(2,5,4) \not \leq (3,1,1)$.

\begin{exercise}
  Consider the infinite partially ordered set $(\nn, \leq)$.
  \begin{enumerate}
  \item
    Which elements are minimal? Which are maximal?
  \item Is there a minimum? A maximum?
  \item Does it have an infinite chain?
  \item Does it have arbitrarily large antichains? That is, can you find an
  antichain $A$ of size $|A| = k$ for every $k \in \N$?
\end{enumerate}
\end{exercise}

\begin{exerciseD}
  Does every infinite subset $S \subseteq \nn$ contain
  an infinite chain?
\end{exerciseD}

\begin{exercise}
  Show that $(\nn,\leq)$ has no infinite antichain. \textbf{Hint.} Use
  the previous exercise.
\end{exercise}



Consider the induced ordering on $\{0,1\}^n$. That is, for $x,y\in \{0,1\}^n$
we have $x \leq y$ if $x_i \leq y_i$ for every coordinate $i \in [n]$.

\begin{exercise}
 Draw the Hasse diagrams of $(\{0,1 \}^n, \leq)$ for $n=2,3$.
\end{exercise}

\begin{exercise}
  Determine the maximum, minimum, maximal, and minimal elements of
  $\{0,1\}^n$.
\end{exercise}

\begin{exercise}
  What is the longest chain of $\{0,1\}^n$?
\end{exercise}

\begin{exerciseDD}
  What is the largest antichain of $\{0,1\}^n$?
\end{exerciseDD}



\subsection{Infinite Sets}

In the lecture (and the lecture notes) we have showed that $\N \times \N \cong \N$, i.e.,
there is a bijection $f: \N \times \N \rightarrow \N$. From this, and by induction, it follows
quite easily that $\N^k \cong \N$ for every $k$.

\begin{exercise}
   Consider $\N^*$, the set of all finite sequences of natural numbers, that is,
   $\N^* = \{\epsilon\} \cup \N \cup \N^2 \cup \N^3 \cup \dots$. Here,
   $\epsilon$ is the empty sequence. Show that $\N \cong \N^*$ by defining
   a bijection $\N \rightarrow \N^*$.
\end{exercise}

\begin{proof}
As we have learn before, $\N \cong \{0,1\}^* $, so we need proof $\N^* \cong \{0,1\)^*$.\\
We could define a function $f$:

$f:\N^* \rightarrow \{0,1\}^*, (a_1, a_2, \dots) \mapsto$ sequnce of 1s with $(a_k + 1)$th bit replaced with 0.\\
In other words, for each element $\mathbf{a} \in \N^*$, its corresponding bit sequence is combination of $a_k$ $1$s appended with a $0$ where $k \in [1, |\mathbf{a}|]$.\\
Since $\{0,1\}^*$ is also a natural number sequence, $|\{0,1\}^*| \le \N^*$. Also, accoding to function $f$, each image of $\mathbf{a} \in \N^*$ in $\{0,1\}^*$ has different construction, which means $\N^* \le |\{0,1\}^*|$. In this way, function $f$ is bijective.
\end{proof}
\begin{exercise}
   Show that $R \cong R \times R$. \textbf{Hint:} Use the fact that
   $R \cong \{0,1\}^{\N}$ and thus show that $\{0,1\}^{\N} \cong \{0,1\}^{\N} \times \{0,1\}^{\N}$.
\end{exercise}

\begin{proof}
We can define $f$:\\
\begin{center}
$f:\{0,1\}^{\N} \times \{0,1\}^{\N} \rightarrow \{0,1\}^{\N}$\\
$((a_{11}, a_{12}, a_{13}, \dots),(a_{21}, a_{22},a_{23},\dots)) \mapsto (a_{11}, a_{21}, a_{12},a_{22}, \dots)$\
\end{center}
For any element $\mathbf{a} \in \{0,1\}^{\N}$, $\mathbf{a}$ can be representated by a combination of unique ordered pair $(\mathbf{a_1}, \mathbf{a_2})$ where $\mathbf{a_1}, \mathbf{a_2} \in \{0,1\}^{\N}$. In this way, $f$ is bijective.\\
Then we can get $R \cong \{0,1\}^{\N} \cong \{0,1\}^{\N} \times \{0,1\}^{\N} \cong R \times R$.\\
\end{proof}

\begin{exercise}
  Consider $\R^{\N}$, the set of all infinite sequences $(r_1, r_2, r_3,\dots)$ of real numbers.
  Show that $\R \cong \R^{\N}$. \textbf{Hint:} Again, use the fact that $\R \cong \{0,1\}^{\N}$.
\end{exercise}

\begin{proof}
Since $R \cong \{0,1\}^{\N}$, we need to prove $(\{0,1\})^{\N})^{\N} \cong \{0,1\}^{\N}$.\\
The element $\mathbf{e} \in (\{0,1\})^{\N})^{\N}$ can be represented as $(\mathbf{a^1},\mathbf{a^2},\dots)$ where $\mathbf{a^i} \in \{0,1\}^{\N}$\\
Also, $\mathbf{e}$ can be represented as a matrix:\\
\begin{equation*}
\mathbf{e} = \left(
\begin{array}{ll}
  \mathbf{a^1}\\
  \mathbf{a^2}\\
  \dots \\
\end{array} \right)
= \left(
\begin{array}{ccc}
  a^1_1 & a^1_2 & \dots\\
  a^2_1 & a^2_2 & \dots\\
  \dots & \dots & \dots\\
\end{array} \right)
\end{equation*}
Define a function $f$:
\begin{center}
$f:\{0,1\} \rightarrow (\{0,1\})^{\N})^{\N}, (a_1, a_2, a_3, \dots) \mapsto \left(
\begin{array}{cccc}
  a_1 & a_3 & a_6 & \dots \\
  a_2 & a_5 & \dots & \dots \\
  a_4 & \dots & \dots & \dots \\
  \dots & \dots & \dots & \dots
\end{array}
\right)$
\end{center}
For $\mathbf{a} \in \{0,1\}^{\N}$, we place each bit of $\mathbf{a}$ on matrix following the diagonals in turn, then we get a unique $\mathbf{e} \in (\{0,1\})^{\N})^{\N}$.\\
Also, for each $\mathbf{e} \in (\{0,1\})^{\N})^{\N}$, there is exactly one element $\mathbf{a} \in \{0,1\}^{\N}$ such that $f(\mathbf{a}) = \mathbf{e}$.\\
In this sense, $\{0,1\} \cong (\{0,1\})^{\N})^{\N}$, which can be expressed as$\R \cong \R^{\N}$.
\end{proof}

Next, let us view $\{0,1\}^{\N}$ as a partial ordering: given two elements $\mathbf{a}, \mathbf{b} \in \{0,1\}^{\N}$,
that is, sequences $\mathbf{a} = (a_1,a_2,\dots)$ and $\mathbf{b} = (b_1,b_2,\dots)$, we define
$\mathbf{a} \leq \mathbf{b}$ if $a_i \leq b_i$ for all $i \in \N$. Clearly,
$(0,0,\dots)$ is the minimum element in this ordering and $(1,1,\dots)$ the maximum.\\

\begin{exercise}
   Give a countably infinite chain in $\{0,1\}^{\N}$. Remember that a set $A$ is countably infinite
   if $A \cong \N$.
 \end{exercise}

\begin{solution}
For set $A \subseteq \{0,1\}^{\N}$, let $A = \{\mathbf{a_i} \in \{0,1\}^{\N} |$ if $k < i$,$a_k =1, else, a_k = 0\}$, any two elements in $A$ are comparable \\
We can define a function $f$:\\
\begin{center}
$f: N \rightarrow A, x \mapsto \mathbf{a_x}$\\
\end{center}
which is obviously bijective. In this way, $A$ is a countably infinite chain.
\end{solution}

\begin{exercise}
   Find a countably infinite antichain in $\{0,1\}^{\N}$.
\end{exercise}

\begin{solution}
For set $A \subseteq \{0,1\}^{\N}$, let $A = \{ \mathbf{a_i} \in \{0,1\}^{\N} | a_i = 1$ and other bits is $0\}$. Any two elements in $A$ is uncomparable since each has one bit larger than the other.\\
Also, we can define a function $f$:\\
\begin{center}
$f: N \rightarrow A, x \mapsto \mathbf{a_x}$\\
\end{center}
which is bijective.\\
In this scense, $A$ is a countably infinite antichain in $\{0,1\}^{\N}$.
\end{solution}

\begin{exercise}
   Find an uncountable antichain in $\{0,1\}^{\N}$. That is, an antichain $A$ with $A \cong \R$.
\end{exercise}

\begin{solution}
Define a function $f$:\\
\begin{center}
$f: \{0,1\}^{\N} \rightarrow A, (a_1, a_2, a_3 \dots) \mapsto (a_1,1-a_1,a_2,1-a_2,\dots)$
\end{center}
$f$ is a bijective function, and $\{0,1\}^{\N} \cong A$\\
For $\mathbf{a},\mathbf{b} \in A$, if $\mathbf{a} \ne \mathbf{b}$ and $\mathbf{a},\mathbf{b}$ are comparable, let $\mathbf{a} < \mathbf{b}$.\\
There must exist $a_i < b_i$, but $ 1-a_i > 1-b_i$, so $\mathbf{a}$ and $ \mathbf{b}$ are uncomparable. Then we get a contradiction.\\
In this way, $A$ is a uncountable antichain.
\end{solution}

\begin{exerciseDD}
   Find an uncountable chain in $\{0,1\}^{\N}$. That is, an antichain $A$ with $A \cong \R$.
\end{exerciseDD}

\end{document}
